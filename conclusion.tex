\chapter{Conclusion}
\label{ch:conclusion}
\acresetall

With this thesis I have shown that the problem of authentication in \aclp{MANET}
can be managed using a special type of public key certificates called proxy
certificates, along with the use of one-time passwords. In this thesis a system
design for handling the authentication in a MANET is proposed along with its
implementation for UNIX systems. I have shown that a central node is only
required for initial authentication, while the subsequent operation of the MANET
have no such requirement due to the use of proxy certificates and a trust
scheme.

Prior to this thesis, a comprehensive study of related works for authentication
and security in ad hoc networks was carried out during a specialization project
in co-operation with Anne Gabrielle Bowitz. During this study the main design
ideas were formed, but changes were made during this thesis to account for
authentication of nodes without a verifiable authentication token, and also
during the implementation when some specific goals became more clear.

An emergency scenario was constructed and used to create system design
requirements. With the exception of one requirement, which has been deemed
further work, all requirements have been accounted for and handled in the
proposed system design.

The implementation of the system design was achieved by extending an existing
and popular ad hoc routing protocol called BATMAN. The extension should also be
portable to other network-layer pro-active routing protocols, with only minor
modifications.

Furthermore, the performance of the implemented system design was compared
to the original routing protocol implementation. These tests indicates that the
extension do not add much delay, but rather a small and constant delay. The
tests are, however not fully definite and they should probably be runned again
with a larger set of test-machines and with far more trials. Only this way can a
linear time delay be disproven. If further testing should prove a constant
delay, there would be no obvious performance issues with the proposed design.

Some tasks are left as future work. As noted in the discussion chapter, both
wormhole attacks and suppress replay attacks might have to be countered. With an
powerful adversary both attacks could be successful on the proposed design,
however two possible solutions (one more or less straight forward) were
discussed. These proposals are left as my suggestions to the science community
for further work. A peer review of the security assumptions and choices in the
proposed design is also left as further work.
