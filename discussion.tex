\chapter{Discussion}
\label{ch:discussion}
\acresetall


\section{Why ECC?}
ECC has short key sizes compared to same strength RSA keys.

RSA have less energy consumption (claimed by \cite{hosseinisecure}) - need to
check it out, or at least mention somewhere\ldots

\section{Routing Authentication Solution}
\label{sect:routing_auth_solution}
As proposed in Chapter \ref{ch:design} the proposed system design uses an novel
solution to continuously verify routing announcements received from ones
neighbors.

For this system to be used on typical mobile devices with all their constraints,
one has to acknowledge the limitations to computing power, battery lifetime, and
saturation in the wireless network.

Because all nodes in a \ac{MANET} using a pro-active routing protocol broadcasts
their routing announcements and forward all received routing announcements, the
network traffic will increase exponentially to the amount of nodes in the
network and how closely bound they are. Therefore all routing announcements
needs to be as small as possible, as discussed by several others before [TODO:
find some refs, multiple]. A typical signature is usually on the order of one or
two larger than a regular routing announcement, so by adding a signature to the
routing announcement - most of the data sent in the network would be signature
data. This is far from ideal.

The first solution that one would think of would be to only sign a very few of
the announcements, periodically. This however, would be totally disastrous. This
would have no protection against spoofing attacks whatsoever, as an attacker
could wait for a legitimate node to send a signed announcement and then send his
own fake announcements spoofed with the legitimate node's address.

The solution proposed in this thesis, solves the problem in a different manner.
Periodically, more than a second and less than a minute, a node sends a special
packet to all its direct neighbors. The packet contains a random value string,
a temporary symmetric key and a signed hash value of the packet (HMAC). Each
copy of the packet is also encrypted with the neighbors' public key in order to
keep the random value and symmetric key hidden from adversaries.

Now, each neighbor shares a secret with the node and this can be used in a trust
scheme on the following routing announcements. The node and its neighbors
computes a new value by encrypting the previously sent random value with the
corresponding symmetric key. This value is then split in parts, or extracts, of
16 bits. Each extract is then appended, along with an offset value declaring at
which offset of the whole value the 16 bits belongs, to all routing
announcements the node sends - both his own and rebroadcasts of other nodes'
announcements. Each offset of 16 bits will only be allowed to be sent once, as
to avoid replay attacks using the same values. 

Neighbors can now easily check whether the extract it receives matches the
correct offset in the value it computed, and if it does the neighbor will trust
the announcement is actually sent by the legitemate node. 

This scheme is fully based on trust. You trust that your trusted nodes will only
send you its own annoucement (correctly) and rebroadcast only its trusted nodes
announcements without modification. If for some reason a trusted node should
behave maliciously, this scheme will not detect this and allow the trusted node
to potentially disrupt the network.

\section{Future Work}

\section{Limitations}

\subsection{IP Address Configuration}\label{ip_address_conf}