\chapter{Discussion}
\label{ch:discussion}
\acresetall

\section{Key Usage}
Not good practive to use same keys for both encryption and signing
Is simpler
Limited lifetime makes it ok\ldots

\section{System Design Vulnerability/Attacks on the routing announcements?}
The design was changed a bit during implementation, and because lack of time
there was a crucial part that did not make it to the implementation, which needs
to be addressed before the system can be used in a real world scenario. There
are two attacks that the system are still vulnerable to, namely the
\emph{wormhole attack} and the \emph{suppress replay attack}.

Because the routing announcements are unencrypted and the one-time-passwords
are just appended after the messages, these packets can be altered using each of
the two attacks. In this section the two attacks are first explained, and then
the section goes on to describe possible solutions for both attacks.

\subsection{Wormhole Attack}
In a wormhole attack, see Figure \ref{fig:wormhole_attack}, an attacker does not
need to know the keystream of a node in order to send that node's routing
announcements. If you look at the figure from the background chapter you have a
network of trusted nodes in green, and two malicious nodes in red. 

Assume the two attackers, M1 and M2, want to disrupt the network topology by
having node B believe node A (and vice versa) is a direct neighbor. Assuming the
out-of-band wormhole in the figure is a faster route between node A and B than
the ``real'' route through the network, M1 can simply forward the announcements
from A to M2. Then M2 does the same, except he also needs to spoof his network
and link layer addresses. When B receives the routing announcement from M2, he
believes it is from A and that A is a new direct neighbor.

Now B will ask for A's keystream-material, and M2 and M1 will forward this
request to A, which will dutifully reply with his keystream-material. Note that
M1 needs to spoof B's addresses here, as well as forwarding B's routing
announcements. When B receives the keystream through the wormhole, he is now
able to verify the routing announcements also sent through the wormhole, and
soon will this route take priority in his routing table as a strong direct link
instead of going through other nodes in the network to reach A.

TODO: MSC FIGURE VISER DETTE

This attack only distorts the network topology, but notice that because the
routing announcements' integrity are not protected, and the content is not
encrypted, the attackers can possibly do greater damage by altering the content
of the message. By altering the content, they could possibly generate fake
re-broadcast announcements which is announcing untrusted nodes, possibly
connecting a whole network of malicious nodes to this trusted network. This
alteration is not actually a wormhole attack, but it is fair to say that the
wormhole attack opens up to a variety of other attacks, such as alteration of
packet, dropping of packets etc.

\subsection{Suppress Replay Attack}
Another attack which is possible due to the fact that the one time passwords are
not ``connected'' to the routing announcements in any way is the suppress replay
attack. Suppose an attacker is able to jam the signals for a very short time,
only enough to distort the main payload of the routing announcement, while the
one time password and its sequence number are kept intact.

The trusted receiver of this distorted packet will ignore it, because he will
not understand the meaning of the packet. The attacker on the other hand, knows
that the following uncorrupt data is a valid one time password and its sequence
number from the sender's keystream.

Because the original recipient of this packet did not understand the destroyed
packet, he will not know that this \ac{OTP} is already used, and if the attacker
wishes to create a false routing announcement he can now do this and append the
\ac{OTP}, which will be accepted by the original recipient. Note that also here
the attacker needs to spoof the addresses of the original sender of the
announcement which he partly jammed.

\subsection{Solution to the Suppress Replay Attack}
A possible solution to the suppress replay attack, is to ensure message
intergrity in addition to the authentication. The basic idea would be to create
and append a message digest to the announcement while using the keystream to
encrypt the message, e.g. by using a strong stream cipher.

There is one great pitfall to this idea, however. Because the routing
announcements are often identical to each other, the attacker might be able to
do a known plaintext attack, being able to first recreate the message digest,
and depending on the stream cipher just XOR the announcement and digest with the
encrypted message, revealing the keystream used by the stream cipher. If this
message was in any way blocked from reaching the intended recipient, the packet
could be changed and XOR'ed with the retrieved keystream and sent to the
recipient, which would trust this packet.

One possible way to avoid this problem would be to add some randomness to the
``known part'', namely the original announcement. If the announcement is e.g.
appended with a part of the senders keystream (\ac{OTP}) before creating the
message digest, the attacker would not be able to recreate the full plaintext
(including digest) and would therefore not be able to find the keystream used by
the stream cipher.

The reader might rise another question, regarding the lenght of this random part
added to the announcement before creating the digest. One major factor here is
time, and time it takes for an adversary to find the random part (collision).
The time is however very restricted. The sliding window used to avoid replay
attacks is only 64 bits long, and in the worst case (best case for adversary)
there are only two legitimate nodes in the network. If this is the case, each
node will send two announcements every second (BATMAN protocol) meaning the
window of opportunity for the attacker is only 32 seconds. For each extra
legitimate node in the network this window drastically closes. The length of the
random part used for the digest should therefore be strong enough withstand 32
seconds of brute-force guessing attack.

Note that it might not be wise to send the whole message digest because the
announcement sizes would become very large. This might be truncated as well, but
when designing this the time needed to find a collision for a truncated digest
needs to be addressed as well.

\subsection{Possible Solution to the Wormhole Attack}
This attack vector is very difficult to to protect against. With the solution
above, many attacks dependent on the wormhole attack are thwarted because of
packet secrecy, integrity and authentication. However, the solution does not
hinder an attacker from replaying the exact same packets through a wormhole, in
order to alter the network topology. There has been alot of research effort to
find a good solution to this attack, but most solutions are aimed towards
stationary networks and not \acp{MANET} \cite{raoteapproaches}.

One possible solution as pointed to by the article above is the use of
``location aware guard node'' and graph theory \cite{poovendran2007graph}
\cite{lazos2005preventing} to detect wormholes. The idea is that if you have
special nodes spread out in the field at fixed points, where none of these nodes
are within each other's transmitting range, one should never be a neighbor of
more than one node at a time. If you receive direct packets from two nodes
within a very short timeframe, there might be indications that there's a
wormhole in place replaying one of the special nodes' announcements.

\section{Extending the System Design}
The specialization project\footnote{} this thesis mentioned other important
features the secure ad hoc network implementation should have. The things
mentioned in this thesis' system design are actually implemented, but there are
still much more that should be added if this system should be used in real life
emergency situations.

In this section some important features which should be added, or at least
studied, are mentioned. They do add more complexity to the system, so it is
probably better to do a complete security and performance analysis on this
thesis' proposed design before adding these features, however.

\subsection{Initial Authentication with Long-Lived Public Key Certificates}
One limiting factor in the system design is the need for an out-of-band
initial authentication. With this limitation, every actor in the emergency
scenario needs to manually verify his or hers identity to the network management
handled by the \ac{SP}. With many actors, which would be typical in a large
emergency situation like a natural disaster, this process might take up valuable
time from the actual emergency work - which contradicts the whole meaning of
setting up the \ac{MANET} at the scene in the first place.

Now, if an actor has possession of a regular \ac{LLPKC} for which the network
\ac{SP} is able to verify, this should be allowed without the need of an
out-of-band authentication. After the \ac{SP} have verified the certificate, he
can now issue the the actor a proxy certificate, signed by him so that all nodes
in the network are able to verify the new nodes identity.

The question of whether the \ac{SP} is able to verify the \ac{LLPKC} is not
necessarily easy to answer. If the \ac{SP} trusts the identity and knows the
public key of the issuer of the actor's certificate, he is able to verify that
this actor was trusted with this identity (and rights) at some time. However, it
does not mean this is true anymore - the certificate might have been revoked.

In the absence of Internet access, \acp{CRL} might not be available to the
\ac{SP}. If this is the case, an evaluation of what to do with the actor has to
be done. Ideas that comes to mind might be to issue a very short lived \ac{PC},
maybe with limited rights, to the actor's node so that it can start working now,
but has to re-authenticate later when \acp{CRL} has been brought to the scene
either out-of-band or by setting up Internet access. If no \ac{CRL} is ever
brought to the SPs attention, he might want to require an out-of-band
authentication later.

Either way, as this does not have to happen in the same out-of-band fashion as
the proposed design in this thesis requires, one could also allow the
authentication happen even if the actor is not a direct neighbor of the \ac{SP},
but is connected through other nodes in the network.

While in this implementation regular trusted nodes drop the announcements from
new unauthenticated nodes, they could rather tunnel the announcement directly to
the \ac{SP} and have the authentication handshake go through them.

\subsection{Network Merging}
\subsection{Internet Gateway}
\subsection{Multiple SP}
\subsubsection*{SP List Sharing}
need to know other SPs public keys, therefore send a list of all SPs with their
public kehys, just like the old AL sending idea, but only for SPs. Each SP keeps
track of their own children nodes, and send them the list, signed with his
pubkey.
