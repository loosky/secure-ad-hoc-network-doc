\chapter{Results}
\label{results}

This section presents the numeric results collected from the tests described in the previous section. The results can be divided in two sections, one for authentication time and one for convergence time.

\section{Authentication Time}

The authentication times calculated for Test I and Test II are shown in Table \ref{tab:auth_time_test_result} along with the minimum, maximum, and average time values.

\begin{table}[ht!]
	\centering
	\begin{tabular}{ | c | l | l | }
	\hline
	\textbf{\#} & \textbf{Test I} & \textbf{Test II}\\ \hline
		1 & 13.30 & 5.82 \\ \hline
		2 & 17.39 & 9.51 \\ \hline
		3 & 8.03 & 15.24 \\ \hline
		4 & 15.46 & 19.10 \\ \hline
		5 & 11.21 & 19.72 \\ \hline
		6 & 12.16 & 7.07 \\ \hline
		7 & 9.13 & 31.03 \\ \hline
		8 & 16.33 & 30.53 \\ \hline
		9 & 17.10 & 14.10 \\ \hline
		10 & 13.99 & 7.28 \\ \hline \hline
		Min & 8.03 & 5.82 \\ \hline
		Max & 17.39 & 31.03 \\ \hline
		Avg & 13.41 & 15.94 \\ \hline
	\end{tabular}
	\caption{Results after 10 runs with Test I and Test II}
	\label{tab:auth_time_test_result}
\end{table}

\noindent
The original BATMAN protocol does of course not have any authentication mechanism thus the first two tests was only run with the modified BATMAN protocol.
\\\\
As we can see from the results above, the authentication process adds significant overhead in the initial phase when establishing new networks or authenticating new nodes. The large difference between the best and worst case can simply be explained as randomness. It is arbitrary how many collisions will occur during an authentication process and size of the back-off time could in the worst case be set to a very high value as this is chosen randomly. 

\section{Convergence Time}
Here we aim to measure route convergence time in three different scenarios and see how our implementation performed against the original BATMAN protocol. Our aim here is mainly to show that once authenticated, BATMAN operation should perform similar to that of the original protocol. Table \ref{tab:conv_time_test_result} shows the results for test III and IV.

\begin{table}[ht!]
	\centering
	\begin{tabular}{ | c | l | l || l | l | }
	\hline
	& \multicolumn{2}{|c||}{\textbf{Test III}} & \multicolumn{2}{c|}{\textbf{Test IV}}\\ \hline
	\textbf{\#} & \textbf{Secure} & \textbf{Original} & \textbf{Secure} & \textbf{Original}\\ \hline
		1 & 18.26 & 20.91 & 41.40 & 29.85 \\ \hline
		2 & 15.02 & 22.14 & 43.04 & 44.02 \\ \hline
		3 & 16.00 & 14.22 & 33.23 & 41.80 \\ \hline
		4 & 14.75 & 13.07 & 48.46 & 44.38 \\ \hline
		5 & 17.94 & 17.45 & 45.82 & 47.28 \\ \hline \hline
		Min & 14.75 & 13.07 & 33.23 & 29.85 \\ \hline
		Max & 18.26 & 22.14 & 48.46 & 47.28 \\ \hline
		Avg & 16.39 & 17.56 & 42.39 & 41.47 \\ \hline
	\end{tabular}
	\caption{Results after 5 runs with Test III and Test IV with both our implementation and the original BATMAN protocol}
	\label{tab:conv_time_test_result}
\end{table}

\noindent
With this table we can produce far more positive results. When it comes to convergence within the network between already authenticated nodes, our implementation does not seem to add any delay. However, as we can see, neither our implementation nor the original implementation performs great. This coincides with previous tests which we mentioned in \ref{batman_olsr_comparison}.

%\section{Statistical Relevance}
%Due to general time constraints, we did not have enough time to conduct all our tests enough times as to get statistically significant results. However, our results are good indications and they should be good enough for this kind of proof of concept.

%We did not get statistically significant results with our tests, because of the high variation and low sample set, but the results seemed to strongly indicate what we already expected so we decided to stop the experiments short.
