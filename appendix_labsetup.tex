\chapter{Lab Setup}
\label{appendix:lab_setup}


The computers used in the lab was setup with the following hardware:

\begin{itemize}
\item Intel Core 2 Duo 2.83 GHz processor
\item 4 GB memory
\item Atheros AR5413 802.11abg NIC
\end{itemize}

\noindent
Further, they are setup with Ubuntu 10.4 (Linux Kernel 2.6.32-25-generic-pae) and ath5k drivers for the wireless interfaces. The network interface is configured as follows:


\begin{lstlisting}[frame=tb]
/etc/network/interfaces

auto lo
iface lo inet loopback

auto wlan0
iface wlan0 inet static
address 10.0.0.X
netmask 255.255.255.0
pre-up ifconfig wlan0 down
pre-up ifconfig wlan0 hw ether XX:XX:XX:XX:XX:XX
pre-up iwconfig wlan0 mode ad-hoc essid BATMAN channel 3

auto unicast
iface unicast inet static
address 10.0.0.X
netmask 255.255.255.0
pre-up brctl addbr unicast
pre-up brctl addif unicast wlan0
pre-down ifconfig unicast down
post-down brctl delif unicast wlan0
post-down brctl delbr unicast
\end{lstlisting}

\noindent
\\
To install batmand, run the following as root user:

\begin{lstlisting}[frame=tb]
make
make install
make clean
\end{lstlisting}

\noindent
\\
For running the batman daemon on the test nodes, we ran the following script as root user:

\begin{lstlisting}[frame=tb]
ifconfig wlan0 up
batmand wlan0
batmand -cd 4
killall batmand
ifconfig wlan0 down
\end{lstlisting}

\noindent
\\
For reducing the transmitting power in test III and IV we ran the following as root user:

\begin{lstlisting}[frame=tb]
ifconfig wlan0 down
iwconfig wlan0 txpower 7
ifconfig wlan0 up
\end{lstlisting}


