\chapter*{Definitions}
\addcontentsline{toc}{chapter}{Definitions}


%TODO: Change to some other package. COnflicts with actual acronyms list.
\begin{acronym}
%Legg inn i alfabetisk rekkefølge!

\acro{802.11}
	IEEE 802.11 standard, wireless, more more

%\acro{Ad Hoc Network}
%		A self-organizing network with no form for pre-existing infrastructure or
%		centralized administration.

\acro{Asymetric Link}
	If traffic is only possible in one direction, i.e. a node can receive but not
	send packets to another node, the link in between them is called an asymetric
	link.

%\acro{Authenticated List} %egendefinert forkortelse av espen
%	A list containing the public keys, IP, roles, certificate validity period,
%	signature fraction and the timestamp of the last received signature of all
%	authenticated nodes in the network. The list broadcasted by the SP
%	periodically.

\acro{Authentication}
	Say something about authentication

\acro{Authentication Module}
	Addition to the B.A.T.M.A.N. protocol which takes care of cryptographic
	functions and other additions. It also adds fields to the Originator messages
	which can contain a digital signature or signature fractions, and sends other
	messages with nonces, certificates, and ALs.

\acro{Authentication Token}
	Say something about authentication tokes, such as certificates and so on\ldots

\acro{Authorization}
	Say something about authorization

\acro{Congestion}
	Congestion is a state in wich the the amount of traffic on a network surpasses
	the stable amount of traffic the network can handle. I.e. congestion can make
	the network useless if not handled by some control mechanisms.
	
\acro{Certificate Authority}
	Say something about CAs.

%\acro{Convergence Time}
%	The time it takes for the network to get to a stable state with no route
%	flapping after an event that has changed the network topology. E.g. a node has
%	died or moved and made a link inferior to other alternative links.

\acro{Elliptic-Curve Cryptography}
	Public key cryptography based on the mathematical properties of elliptic
	curves.

\acro{End-Entity Certificate}
	A X.509 public key certificate of an end user.

\acro{Link-local}
	See Neighbor.

%\acro{Multicast}
%	In computer networking this refers to the delivery of a packet or message to a
%	group of devices.

\acro{Neighbor}
	Neighbor refer to actual link-local a neighbor, i.e. a node within
	transmitting range for which you can communicate directly with.

\acro{OASIS}
	Trengs dette?

%\acro{Originator}
%	Synonym for a Batman interface which is a network interface utilized by
%	Batman.

%\acro {Originator Message}
%	Batman protocol message advertising the existence of an originator. They are
%	used for link quality and path detection \cite{batman_rfc}. %CP

%\acro{Packet Delivery Ratio}
%	Proportion of delivered packets relative to the amount of packets sent.

\acro{Pro-Active Routing}
	TODO

\acro{Proxy Certificate} 
	A X.509 certificate signed by a regular X.509 EEC. It is used to assign roles
	to which the recipient can act on behalf of the signee.

\acro{Proxy Certificate 0}
	Say something about PC0
	
\acro{Proxy Certificate 1}
	Say something about PC1
	
	
\acro{Public Key Infrastructure}
	Every entities involved with the management (creation, distribution etc.) of
	public key certificates. Managed by the PKIX working group of IETF.

%\acro{Round Trip Delay}
%	The time it takes from a packet is sent from the sender and the sender
%	receives as acknowledgment packet from the receiver.

\acro{Route Flapping}
	Occurs when a node in a network continuously changes preferred route between a
	source and destination pair creating route instability.


\acro{Routing Protocol}
	TODO

\acro{Service Proxy}
	Say something about SPs.

%\acro{Shortest Path}
%	Minimum number of hops between two communicating nodes.

\acro{Socket}
	Say something about sockets.

\acro{Thread}
	Say something?

\acro{Web Of Trust}
	GnuPG project\ldots

\acro{X.509 Certificates}
	Standard public key certificate standard managed by the PKIX working group of
	IETF.

\end{acronym}
