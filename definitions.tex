\chapter*{Definitions}
\addcontentsline{toc}{chapter}{Definitions}


\begin{acronym}

\acro{802.11}
	IEEE 802.11 standards for wireless computer networks on the 2.4, 3.6, and 5 GHz
	frequency bands.

\acro{Ad Hoc Network}
	A self-organizing network with no form for pre-existing infrastructure or
	centralized administration.

\acro{Asymmetric Link}
	If traffic is only possible in one direction, i.e. a node can receive but not
	send packets to another node, the link in between them is called an asymetric
	link.

\acro{Authenticated List} %egendefinert forkortelse av espen
	A list containing the public keys, IP, roles, certificate validity period,
	of all known and authenticated nodes in the network.

\acro{Authentication}
	The process of verifying an alleged identification.

\acro{Authentication Module}
	Addition to the B.A.T.M.A.N. protocol which takes care of cryptographic
	functions and other additions. It also adds fields to the Originator messages
	which can contain a digital signature or signature fractions, and sends other
	messages with nonces, certificates, and ALs.

\acro{Authentication Token}
	Is something that can help the authentication process. This could be e.g. a
	public key certificate or a smart card etc.

\acro{Authorization}
	The process of deciding which rights, or access to which resources, an
	authenticated identity can have.

\acro{Certificate Authority}
	An entity that issues certificates in a Public Key Infrastructure (PKI).


\acro{Congestion}
	Congestion is a state in wich the the amount of traffic on a network surpasses
	the stable amount of traffic the network can handle. I.e. congestion can make
	the network useless if not handled by some control mechanisms.
	
\acro{Convergence Time}
	The time it takes for the network to get to a stable state with no route
	flapping after an event that has changed the network topology. E.g. a node has
	died or moved and made a link inferior to other alternative links.

\acro{Elliptic-Curve Cryptography}
	Public key cryptography based on the mathematical properties of elliptic
	curves.

\acro{End-Entity Certificate}
	A X.509 public key certificate of an end user.
	
\acro{Ephemeral Key}
	A temporary symmetric encryption key. 

\acro{Keystream-material message}
	A message containing an encrypted ephemeral key, IV, nonce, and a digital
	signature used to generate the sender's keystream.

\acro{Link-local}
	See Neighbor.

\acro{Neighbor}
	Neighbor refer to actual link-local a neighbor, i.e. a node within
	transmitting range for which you can communicate directly with.

\acro{Originator}
	Synonym for a Batman interface which is a network interface utilized by
	Batman.

\acro {Originator Message}
	Batman protocol message advertising the existence of an originator. They are
	used for link quality and path detection \cite{batman_rfc}.

\acro{Packet Delivery Ratio}
	Amount of packets received divided by the number of packets sent.

\acro{Pro-Active Routing Protocol}
	A routing protocol that regularly broadcasts routing announcements and
	forwards routing announcements it receives, actively discovering routes before
	(pro-active) they are needed for data traffic.

\acro{Proxy Certificate} 
	A X.509 certificate signed by a regular X.509 EEC. It is used to assign roles
	to which the recipient can act on behalf of the signee.

\acro{Proxy Certificate 0}
	A proxy certificate belonging to a Service Proxy, able to issue new proxy
	certificates (PC1), delegating its rights to the receiver of that proxy
	certificate.
	
\acro{Proxy Certificate 1}
	A proxy certificate belonging to regular trusted nodes, not able to issue other
	proxy certificates.
	
	
\acro{Public Key Infrastructure}
	Every entities involved with the management (creation, distribution etc.) of
	public key certificates. Managed by the PKIX working group of IETF.

%\acro{Round Trip Delay}
%	The time it takes from a packet is sent from the sender and the sender
%	receives as acknowledgment packet from the receiver.

\acro{Route Flapping}
	Occurs when a node in a network continuously changes preferred route between a
	source and destination pair creating route instability.


\acro{Routing Protocol}
	A protocol that finds and creates paths, or routes, to other nodes in the
	network.

\acro{Service Proxy}
	Master node controlling the authentication and authorization functions in
	the network. The SP has a PC0 issued by its regular long-lived public key
	certificate.

%\acro{Shortest Path}
%	Minimum number of hops between two communicating nodes.

\acro{Socket}
	UNIX file descriptors or logical interfaces used to send and receive data over
	a network interface.

\acro{Thread}
	A separated program flow sharing memory with its parent process. Used to
	utilize multiple CPUs or CPU cores, or to have more than one thread running
	tasks at the same time.

\acro{Web Of Trust}
	A decentralized trust model where trust of a node is established if your
	trustees trusts that node.

\acro{X.509 Certificates}
	Standard public key certificate standard managed by the PKIX working group of
	IETF.

\end{acronym}
