\chapter{Introduction}
\label{ch:intro}
\acresetall

Today, we have become accustomed with an almost complete presence of digital
networks in our daily lives. Everywhere we go, you can either plug your laptop
into an ethernet slot, connect your laptop to an available wifi hot spot, or
just use your cell phone via 3G mobile data network. However, this is not
universally true throughout the world. Many places are scarcely populated, or
the people living there do not have the resources to deploy such networks.

In emergency and/or military situations, this often applies. Even if it didn't,
the networks may have been put out of operations due to the nature of the
emergency (i.e. tsunami destroying the infrastructure). The military might
also be in an hostile environment where they cannot trust the network in place
altogether.

Emergency search and rescue and military tactical operations can greatly benefit
from the use of digital communication for sharing operation critical
information. If they have no trusted data network available, they should
therefore setup one themselves. A realistic approach would have to be easy and
quick to setup and be self-managing thus requiring minimal maintenance. It
should to an extent always be available to the participants wherever they go,
which calls for using a wireless network. Last but not least, the network needs
to be trusted, i.e. you should trust that the infrastructure is not compromised
and that the communicating parties on the network are who they claim to be.

A \ac{MANET} solves some of these requirements. It does not need an existing
communication infrastructure, it is self-organizing and the network coverage
range can easily be extended by placing intermediate nodes in strategic
locations. The latter requirement however, is a more challenging task in
\acp{MANET}.

With the lack of infrastructure in \acp{MANET} and no guarantees they are
connected to the Internet, establishing trust between the nodes become different
from that on the Internet which can rely on e.g. \acp{PKI}.

In this thesis I will propose and implement a solution suggestion to establish a
trust mechanism, i.e. an authentication scheme, which combines features of a
typical \ac{PKI} with \ac{WOT} ideas [TODO: ref of gpg project maybe?].


\section{Motivation}
The motivation for research in this field is very high because the possible
social benefits are great. For instance, the recent 7.0 magnitude earthquake
that struck Haiti in 2010 [TODO: REF it? News-source or scientific?]
[Wired-article \url{http://www.wired.com/magazine/2010/04/ff_haiti/} ?] showed
us how huge relief efforts easily become very inefficient when huge amounts of
emergency relief personell work at a scene with little or scarce communication
throughout the area. With a trusted communication network like a secure
\ac{MANET} an operation like this could become much more efficient, bringing the
right amount of help to the right places at the right time.

\section{Contributions?}
Specify what I am contributing with?

\section{Objectives}

\section{Limitations}

\section{Method}
The primary research method conducted in this thesis is the design science
paradigm for Information Systems research as described in [TODO: Hevner,
March\ldots]. The model and method artifacts of this paradigm are described in
Chapter \ref{ch:design} whereas the instantiation artifact is described in
Chapter \ref{ch:implementation}.

\section{Document Structure}
This thesis report is structured as follows:

\textbf{Chapter \ref{ch:background}: Background} aims to give the reader the
necessary insight about the technologies, ideas and theories discussed later in
this thesis.

\textbf{Chapter \ref{ch:design}: System Design} propose an original solution
for an authentication scheme for \acp{MANET}.

\textbf{Chapter \ref{ch:implementation}: Implementation} presents the
implementation of the system design. The implementation is a modification of the
\ac{BATMAN} source code.

\textbf{Chapter \ref{ch:testing}: Testing} devise different tests for checking
the performance of the implementation compared to the original \ac{BATMAN}
implementation, and how well the implementation withstands some known \ac{MANET}
cryptographic attacks.

\textbf{Chapter \ref{ch:results}: Results} presents the results of the tests
from the previous chapter.

\textbf{Chapter \ref{ch:discussion}: Discussion}

\textbf{Chapter \ref{ch:conclusion}: Conclusion}

\textbf{Appendix \ref{appendix:source}: Source Code}

\textbf{Appendix \ref{appendix:irc}: IRC Chat Logs}