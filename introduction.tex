\chapter{Introduction}
\label{ch:intro}
\acresetall
We have become accustomed to an almost complete presence of digital networks
in our daily lives. Everywhere you go, you can either plug your laptop into an
ethernet slot, connect your iPad to an available wifi hot spot, or just use
your cell phone via 3G mobile data network. However, this is not universally
true throughout the world. Many places are sparsely populated, or the people
living there do not have the resources to deploy such networks.

In emergency and/or military situations, this often applies. Even if it didn't,
the networks may have been put out of operations due to the nature of the
emergency (i.e. tsunami destroying the infrastructure). As recent events
here in Norway have shown internal errors might paralyze the whole network 
infratrucutre\footnote{\url{http://www.dagbladet.no/2011/06/16/nyheter/innenriks/telenor/16942385/}
(Norwegian)} making emergency relief ineffective\footnote{\url{http://www.dagbladet.no/2011/06/11/nyheter/ver/flom/naturkatastrofer/innenriks/16880835/}
(Norwegian)}, giving a sound argument for having a separate backup emergency
network. The military might also be in an hostile environment where they cannot trust
the network in place altogether.

Emergency search and rescue and military tactical operations can greatly benefit
from the use of digital communication for sharing operation critical
information. If they have no trusted data network available, they should
therefore set up one themselves. A realistic approach would have to be easy and
quick to set up and be self-managing, thus requiring minimal maintenance. It
should to an extent always be available to the participants wherever they go,
which calls for using a wireless network. Last but not least, the network needs
to be trusted, i.e. you should trust that the infrastructure is not compromised
and that the communicating parties on the network are who they claim to be.

A \ac{MANET} solves some of these requirements. It does not need an existing
communication infrastructure, it is self-organizing and the network coverage
range can easily be extended by placing intermediate nodes in strategic
locations. The latter requirement however, is a more challenging task in
\acp{MANET}.

With the lack of infrastructure in \acp{MANET} and no guarantees that they are
connected to the Internet, establishing trust between the nodes becomes
different from how this is done on the Internet which can rely on e.g.
\acp{PKI}.

In this thesis I will propose and implement a solution suggestion to establish a
trust mechanism, i.e. an authentication scheme, which combines features of a
typical \ac{PKI} with some of the ideas behind \ac{WOT}
\cite{zimmermann1995official}. The system design is presented in two parts, the
part which has been implemented in Chapter \ref{ch:design}, and the ideas that I
did not have time to implement is discussed as further work in Chapter
\ref{ch:discussion}.

As one might expect, it is a very challenging task to achieve strong security
for \acp{MANET} and still have the benefits of its simple ``plug and play''
design. As real world implementations go, there are a few trade-offs, and
security cannot always win. This design will not try to be 100\% secure, but
should be secure enough to deploy in emergency situations. To back this claim,
the Norwegian Army recently stated that their new computer security guidelines
is to rather have a usable (available) system which might be open to attack,
instead of a bad system which is impenetrable - as long as they are able to
monitor and take action against potential
attacks\footnote{\url{http://www.tu.no/it/article287598.ece} (Norwegian)}.

\section{Motivation}
The 7.0 magnitude earthquake that struck Haiti in 2010 showed us how huge
relief efforts easily become very inefficient when huge amounts of emergency
relief personell work at a scene with little or scarce communication throughout
the area\footnote{\url{http://www.wired.com/magazine/2010/04/ff_haiti/}}. With
a trusted communication network like a secure \ac{MANET} an operation like this
could become much more efficient, bringing the right amount of help to the
right places at the right time.

\section{Contributions}
This thesis produce a novel design method to achieve authentication and trust
between nodes in a secure ad hoc network. The popular ad hoc routing protocol
called BATMAN has been extended to become an instantiation of said design, which
has never been done before. Additionally, the use of proxy certificates for
trust establishment for ad hoc networks is also a novel approach to the problem.

\section{Objectives}
The main objective of this thesis is to design and implement an authentication
extension to ad hoc networks based on a known routing protocol.

Secondly, other design ideas, or things that was supposed to be in the design
but did not make the timeframe is discussed upon in contexts of both security
and real world performance.

Last, but not least, testing of the proposed design's implementation should be
done to compare the performance of the new implementation against the original
routing protocol.

The problem description also mentions testing the implementation against known
security attacks. However, my responsible professor Stig Frode Mj{\o}lsnes
claimed no such tests were necessary as the security of this design should
rather undergo peer review and testing, therefore these tests have not been
done.

\section{Limitations}

\subsection{IP Address Configuration}
\label{limit:ip_address_conf}
Autoconfiguration of network interfaces for ad hoc networks is a huge and
difficult task and will not be addressed in this thesis. Throughout this thesis
the assumption is that all nodes trying to participate in the same network is
preconfigured with a valid and unique IP on the correct subnet. It is also
assumed their network interfaces are correctly set up to connect to the correct
wireless channels.

\subsection{Detecting malicious behaviour}
\label{limit:malicious_behaviour}
One attack vector which will not be discussed in this thesis is if a legitimate
node acts maliciously, which might happen if a legitimate node is compromised.
The solution proposed in the thesis assumes all trusted nodes acts with good
intentions. There are much research about detecting malicious behavior in ad
hoc network \cite{Pirzada_McDonald} \cite{dhurandher2010network}.

However, these kind of solutions are mainly designed for networks without any
authentication scheme at all, and is therefore just investigating malicious
behaviour without trying to detect whether a node is compromised or not. These
proposals might therefore not be of the greatest interest, but should be studied
to see if any of their features can safely be applied to an ad hoc network with
an authentication system in place.

\section{Method}
The primary research method conducted in this thesis is the \emph{design
science paradigm} for Information Systems research as described in
\cite{hevner2003information}. The model and method artifacts of this paradigm
are described in Chapter \ref{ch:design} whereas the instantiation artifact is
described in Chapter \ref{ch:implementation}.

Much of the design (method artifact) comes from the specialization project last
fall\footnote{A.G. Bowitz and E.G. Graarud \textit{Developing a Secure Ad Hoc
Network Implementation} \url{http://org.ntnu.no/batman2010/project_final.pdf}},
but some aspects of that design has been changed during the course of the study
and implementation in this thesis.

\section{Document Structure}
This thesis report is structured as follows:

\textbf{Chapter \ref{ch:background}: Background} aims to give the reader the
necessary insight about the technologies, ideas and theories discussed later in
this thesis.

\textbf{Chapter \ref{ch:design}: System Design} proposes an original solution
for an authentication scheme for \acp{MANET}.

\textbf{Chapter \ref{ch:implementation}: Implementation} presents the
implementation of the system design. The implementation is a modification of the
\ac{BATMAN} source code.

\textbf{Chapter \ref{ch:testing_results}: Testing \& Results} devise different
tests for checking t he performance of the implementation compared to the
original \ac{BATMAN} implementation, and presents the results of the tests.

\textbf{Chapter \ref{ch:discussion}: Discussion}

\textbf{Chapter \ref{ch:conclusion}: Conclusion}

\textbf{Appendix \ref{appendix:source}: Source Code}

\textbf{Appendix \ref{appendix:irc}: IRC Chat Logs}