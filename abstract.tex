\chapter*{Abstract}
\addcontentsline{toc}{chapter}{Abstract}

%In emergency situations and military operations it is useful to be able to
%quickly establish communication. It is often necessary to accomplish this with
%minimum pre-existing infrastructure and without centralized administration. In
%such scenarios it would also be important that the network is secure - not only
%implying keeping the communication secret, but also be able to restrict access
%to the network. Wireless ad hoc networks fulfill many of these requirements,
%but the issue of security and access control still remains a challenging task.
%\\\\

In emergency situations such as natural disasters the emergency personell should
be able to establish communication fast and reliably. Depending on the nature of
the disaster one cannot rely on existing communication infrastructure, or access
to centralized administration. Additionally the established communication needs
authentication in order to handle access control so only trusted parties can
partake. A suitable medium for such communication is wireless ad hoc networks,
but their flat structure make authentication a very challenging task.

%The goal of this study can be divided into two parts. The first part was
%focused on trying to define a system with a proper authentication scheme that
%does not affect the nature of ad hoc networks. We combined common
%authentication mechanisms and an ad hoc routing protocol for this purpose.
%Secondly, the B.A.T.M.A.N. routing protocol was extended to incorporate the
%very basic functionality of the system design proposed.
%\\\\

In this thesis a system design for an ad hoc network combined with access
control is proposed, and implemted extending a popular routing protocol called
BATMAN. The proposed authentication scheme relies on special public key
certificates called proxy certificates, and in combination with a neighbor trust
mechanism both authentication and access control are managed in a secure manner.

%A small laboratory environment was set up to test the performance of the
%extended protocol with the intention of proving that our basic functionality
%did not weaken the unique properties of mobile ad hoc networks. The test
%results shows that the basic idea of our system design is possible, and that
%the current implementation should be further extended to fulfill the
%requirements necessary for a secure ad hoc network.

Tests using mobile nodes shows that the performance of the proposed design is
comparable to the original routing protocol (BATMAN) used, showing that the
authentication process is completely manageable even in mobile ad hoc networks.