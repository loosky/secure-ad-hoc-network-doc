\chapter{Source Code}
\label{appendix:source}
\acresetall


\section{Code Snippets}
In this section all the code snippets referred to from Chapter
\ref{ch:implementation} are shown. They might not reflect the source code
perfectly, as some lines in between and re-organization has been done to only
include the most important steps. However, all the values are the exact same as
their counterparts in the source code.

\subsection{AM Sockets Setup}\label{code:sockets}
\begin{lstlisting}[frame=tb]
int32_t *recvsock, *sendsock;
addrinfo hints, *res;

/* Set family information */
memset(&hints, 0, sizeof hints);
hints.ai_family = AF_INET;
hints.ai_socktype = SOCK_DGRAM;
hints.ai_flags = AI_PASSIVE;
hints.ai_protocol = IPPROTO_UDP;

/* Puts the port-info inside a addrinfo data structure */
getaddrinfo(NULL, port, &hints, &res);

/* Assign file descriptor for sockets */
*recvsock = socket(PF_INET, SOCK_DGRAM, 0)
*sendsock = socket(PF_INET, SOCK_DGRAM, 0)

/* Binds the sockets to the network interface */
setsockopt(*recvsock, SOL_SOCKET, SO_BINDTODEVICE, interface, strlen(interface) + 1)
setsockopt(*sendsock, SOL_SOCKET, SO_BINDTODEVICE, interface, strlen(interface) + 1)

/* Binds receive socket to the port (rest of the address is empty/null) */
bind(*recvsock, res->ai_addr, res->ai_addrlen);

/* Allow the send socket to send broadcast messages */
int broadcast_val = 1;
setsockopt(*sendsock, SOL_SOCKET, SO_BROADCAST, &broadcast_val, sizeof int)

/* Set the send socket to non-blocking */
fcntl(*sendsock, F_SETFL, O_NONBLOCK);

\end{lstlisting}

\subsection{Proxy Certificate Extension}\label{code:pc_ext}
Remember the author did not get the original proxyCertInfoExtension to work, so
it was changed with a netscape comment.

\begin{lstlisting}[frame=tb]
STACK_OF(X509_EXTENSION) *exts = sk_X509_EXTENSION_new_null();
openssl_cert_add_ext_req(exts, NID_netscape_comment, "critical,myProxyCertInfoExtension:0,1");

X509_REQ_add_extensions(x, exts);
sk_X509_EXTENSION_pop_free(exts, X509_EXTENSION_free);
\end{lstlisting}

The first digit with a 0 in the comment tells the role the node requests is only
regular 'authenticated'. If this was  1 it would mean the node was a SP. The
last digit is used for routing rights. The 1 in the last digit means the node
wishes full routing rights, whereas a 0 would mean limited routing rights.

\subsection{Setting Subject Name in PC}\label{code:set_subject_name}
\begin{lstlisting}[frame=tb]
X509_NAME *name, *req_name, *issuer_name;
req_name = X509_REQ_get_subject_name(req)
issuer_name = X509_get_subject_name(*pc0p)
name = X509_NAME_dup(issuer_name)
req_name_entry = X509_NAME_get_entry(req_name,0);
X509_NAME_add_entry(name, req_name_entry, X509_NAME_entry_count(name), 0);
X509_set_subject_name(cert, name)
\end{lstlisting}

\subsection{Adding Trusted Node to AL}\label{code:add_to_al}
\begin{lstlisting}[frame=tb]
void al_add(uint32_t addr, uint16_t id, role_type role, unsigned char *subject_name, EVP_PKEY *key) {

	authenticated_list[num_auth_nodes] = malloc(sizeof(trusted_node));
	authenticated_list[num_auth_nodes]->addr = addr;
	authenticated_list[num_auth_nodes]->id = id;
	authenticated_list[num_auth_nodes]->role = role;
	authenticated_list[num_auth_nodes]->name = malloc(FULL_SUB_NM_SZ);
	memset(authenticated_list[num_auth_nodes]->name, 0, FULL_SUB_NM_SZ);

	if(strlen((char *)subject_name)>FULL_SUB_NM_SZ)
		memcpy(authenticated_list[num_auth_nodes]->name, subject_name, FULL_SUB_NM_SZ);
	else
		memcpy(authenticated_list[num_auth_nodes]->name, subject_name, strlen((char *)subject_name));

	authenticated_list[num_auth_nodes]->pub_key = openssl_key_copy(key);

	if(id != my_id) {
		EVP_PKEY_free(key);
	}

	num_auth_nodes++;

}
\end{lstlisting}

\subsection{Adding Trusted Neighbor to NL}\label{code:add_to_nl}
\begin{lstlisting}[frame=tb]
void neigh_list_add(uint32_t addr, uint16_t id, unsigned char *mac_value) {

	int i;
	for(i=0; i<num_trusted_neigh; i++) {
		if(id == neigh_list[i]->id) {

			if(addr == neigh_list[i]->addr) {

				if(neigh_list[i]->mac != NULL)
					free(neigh_list[i]->mac);

				neigh_list[i]->mac = mac_value;
				neigh_list[i]->window = 0;
				neigh_list[i]->last_seq_num = 0;
				neigh_list[i]->last_rcvd_time = time (NULL);
				neigh_list[i]->num_keystream_fails = 0;

			} else {

				if (mac_value != NULL)
					free(mac_value);

				neig_list_remove(i);
				
			}
			
			break;
			
		}
		
	}

	if(i==num_trusted_neigh) {

		neigh_list[num_trusted_neigh] = malloc(sizeof(trusted_neigh));
		neigh_list[num_trusted_neigh]->addr = addr;
		neigh_list[num_trusted_neigh]->id = id;
		neigh_list[num_trusted_neigh]->mac = mac_value;
		neigh_list[i]->window = 0;
		neigh_list[num_trusted_neigh]->last_seq_num = 0;
		neigh_list[num_trusted_neigh]->last_rcvd_time = time (NULL);
		neigh_list[num_trusted_neigh]->num_keystream_fails = 0;
		num_trusted_neigh++;

	}

}

\end{lstlisting}

\subsection{Removing Trusted Neighbor to NL}\label{code:rem_from_nl}
\begin{lstlisting}[frame=tb]
int neig_list_remove(int pos) {

	/* First check whether this node exists at all (sanity check) */
	if(neigh_list[pos] == NULL) {
		return 0;
	}

	/* Check whether keystream exists, remove if so! */
	if(neigh_list[pos]->mac != NULL)
		free(neigh_list[pos]->mac);

	/* Free up neighbor in memory */
	free(neigh_list[pos]);

	/* Re-arrange Neighbor List to avoid scarce population */
	int i;
	for(i=pos+1; i<num_trusted_neigh; i++) {
		neigh_list[i-1] = neigh_list[i];
	}

	/* Finally, number of trusted neighbors has shrunk :) */
	num_trusted_neigh--;

	return 1;
}
\end{lstlisting}

\subsection{Generate Ephemeral Key}\label{code:gen_eph_key}
\begin{lstlisting}[frame=tb]
void openssl_key_generate(EVP_CIPHER_CTX *aes_master, int key_count, unsigned char **keyp) {

	unsigned char *ret;
	int i, tmp, ol;

	if(keyp == NULL || *keyp == NULL) {
		ret = malloc(EVP_CIPHER_CTX_block_size(aes_master));
	} else {
		memset(*keyp, 0, EVP_CIPHER_CTX_block_size(aes_master));
		ret = *keyp;
	}

	ol = 0;

	/* Create plaintext from key_count - each new key will be cipher of i=1,2,3... */
	unsigned char *plaintext = malloc(sizeof(key_count));
	memset(plaintext, 0, sizeof(plaintext));
	*plaintext = (unsigned char)key_count;
	int len = strlen((char *)plaintext)+1;

	EVP_EncryptUpdate(aes_master, ret, &tmp, plaintext, len);
	ol += tmp;
	//Remove padding, not wanted for key!
	EVP_EncryptFinal(aes_master, ret, &tmp);

	free(plaintext);
	*keyp = ret;

}
\end{lstlisting}


\subsection{Generate Keystream}\label{code:gen_keystream}
\begin{lstlisting}[frame=tb]
/* Generate Keystream from Nonce */

if(*key_count>1)
	free(auth_value);

int rand_len = RAND_LEN;
auth_value = malloc(rand_len*10+10);
auth_value_len = 0;

for(i=0; i<10; i++) {

	/* Do encryption */
	EVP_CIPHER_CTX current_ctx;
	EVP_EncryptInit(&current_ctx, EVP_aes_128_cbc(), current_key, current_iv);
	unsigned char *tmp = openssl_aes_encrypt(&current_ctx, current_rand, &value_len);
	EVP_CIPHER_CTX_cleanup(&current_ctx);

	/* Place ciphertext in keystream */
	int auth_pos = auth_value_len;
	auth_value_len += value_len;
	memcpy(auth_value+auth_pos, tmp, value_len);

	/* Change to new IV */
	memcpy(current_iv, tmp, AES_IV_SIZE);

	/* Alter the Nonce before next encryption */
	int j;
	for(j=0;j<rand_len/10; j++) {
		current_rand[j+(i*(rand_len/10))] = ( (current_rand[j+(i*(rand_len/10))]) ^ i );
	}
	
	free(tmp);
	value_len = RAND_LEN;

}
\end{lstlisting}

\subsection{Extension in BATMAN Class}\label{code:ext_batman}
\begin{lstlisting}[frame=tb]
/******************** Begin Authentication Module Extension ********************/

/*
 * If the daemon is not authenticated, or it receives an authentication
 * token which it does not recognize, the authentication procedure in the
 * Authentication Module is called. No packets received when authenticating
 * will be processed.
 */

if(num_trusted_neigh) {
	for(neigh_counter = 0; neigh_counter < num_trusted_neigh; neigh_counter++) {
		if(neigh_list[neigh_counter]->addr == neigh) {
			break;
		}
	}
}

if(neigh_counter == num_trusted_neigh) {

	if(my_role == SP && my_state ==  READY) {
		new_neighbor = neigh;
	}

	if(my_role == AUTHENTICATED && my_state ==  READY) {
		/* Check to see whether the other node is AUTHENTICATED */
		if(memcmp(&(bat_packet->auth), empty_check, 2) != 0)
			new_neighbor = neigh;
	}

	goto send_packets;
}

if(neigh_list[neigh_counter]->mac == NULL)
	goto send_packets;

if(memcmp(neigh_list[neigh_counter]->mac+(bat_packet->auth_seqno*2), bat_packet->auth, 2) != 0) {

	printf("MAC Extract did not match!\n");

	if(my_state == READY) {

		neigh_list[neigh_counter]->num_keystream_fails ++;

		/* Keystream is consequently fail, ergo need to handshake a new one */
		if(neigh_list[neigh_counter]->num_keystream_fails > 20) {
			my_state = WAIT_FOR_REQ_SIG;
			new_neighbor = neigh;
			neigh_list[neigh_counter]->num_keystream_fails = 0;
		}

	}

	goto send_packets;
}

/* Check whether the packet is new and not a replayed packet */
if(!tool_sliding_window(bat_packet->auth_seqno, neigh_list[neigh_counter]->id))
	goto send_packets;


/* Everything seems fine, reset failcounter if more than 0 */
if(neigh_list[neigh_counter]->num_keystream_fails != 0)
	neigh_list[neigh_counter]->num_keystream_fails = 0;

/********************* End Authentication Module Extension *********************/
\end{lstlisting}

\subsection{Extension in SCHEDULE Class}\label{code:ext_schedule}
\begin{lstlisting}[frame=tb]
/* Begin Authentication Module Extension */

/* Add Signature Extract to OGM */
if(pthread_mutex_trylock(&auth_lock) == 0) {

	if(auth_value != NULL) {

		memcpy(bat_packet->auth, auth_value+2*auth_seq_num, 2);
		bat_packet->auth_seqno = auth_seq_num;
		auth_seq_num ++;

	}

	pthread_mutex_unlock(&auth_lock);
}

/* End Authentication Module Extension */
\end{lstlisting}

%\section{BATMAN}
%
%\subsection{batman.h - struct bat\_packet}
%\lstinputlisting[frame=tb]{source_code/batman.h}
%
%\subsection{batman.c - batman()}
%\lstinputlisting[frame=tb]{source_code/batman.c}
%
%
%\section{SCHEDULE}
%
%\subsection{schedule.c - excerpt}
%Line numbers indicate where the code is added to the original source code.
%\lstinputlisting[frame=tb]{source_code/schedule.c}
%
%
%\section{AM}
%
%\subsection{am.h}
%\lstinputlisting[frame=tb]{source_code/am.h}
%
%\subsection{am.c}
%\lstinputlisting[frame=tb]{source_code/am.c}
