\chapter{Source Code}
\label{source_code}

\definecolor{darkgray}{rgb}{0.95,0.95,0.95}
\lstset{ %
language=c,                % choose the language of the code
basicstyle=\footnotesize,       % the size of the fonts that are used for the code
numbers=left,                   % where to put the line-numbers
numberstyle=\footnotesize,      % the size of the fonts that are used for the line-numbers
stepnumber=0,                   % the step between two line-numbers. If it is 1 each line will be numbered
numbersep=5pt,                  % how far the line-numbers are from the code
backgroundcolor=\color{darkgray},  % choose the background color. You must add \usepackage{color}
showspaces=false,               % show spaces adding particular underscores
showstringspaces=false,         % underline spaces within strings
showtabs=false,                 % show tabs within strings adding particular underscores
frame=single,   		% adds a frame around the code
tabsize=2,  		% sets default tabsize to 2 spaces
captionpos=b,   		% sets the caption-position to bottom
breaklines=true,    	% sets automatic line breaking
breakatwhitespace=false,    % sets if automatic breaks should only happen at whitespace
escapeinside={\%*}{*)}          % if you want to add a comment within your code
}

\section{BATMAN}

\subsection{batman.h - struct bat\_packet}
\lstinputlisting[frame=tb]{source_code/batman.h}

\subsection{batman.c - batman()}
\lstinputlisting[frame=tb]{source_code/batman.c}


\section{SCHEDULE}

\subsection{schedule.c - excerpt}
Line numbers indicate where the code is added to the original source code.
\lstinputlisting[frame=tb]{source_code/schedule.c}


\section{AM}

\subsection{am.h}
\lstinputlisting[frame=tb]{source_code/am.h}

\subsection{am.c}
\lstinputlisting[frame=tb]{source_code/am.c}
