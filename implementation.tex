\chapter{Implementation}\label{chapter_implementation}

This chapter goes into depth on how the design from the previous chapter got
implemented into the BATMAN protocol. The layout of this chapter is a
chronological order from the point of view of the programmer. If you wish to
have a look at the code right away, you can find that in Appendix
\ref{chapter_source_code}.


\section{Classes}

\subsection{Authentication Module}
Almost all functionality added to BATMAN is within the borders of this class.
Some code however, had to be added elsewhere in order to call the \ac{AM}
functions, or to use information from \ac{AM}. This segregation is primarily
done in order to separate original BATMAN functions from new authentication
functions.

The first thing to notice about the \ac{AM} is that it runs on its own thread,
so all authentication mechanisms run concurrent to regular BATMAN routing
operations. This separation was necessary in order to have BATMAN behave
normally during authentication of a node, so a large network should not suffer
if an important node (position) is 'hung up' in authentication.

\subsection{Batman and Schedule Classes}
These were the only classes of the original BATMAN code that had to be modified.
The batman class handles most of the logic when receiving a \ac{OGM} so this is
where a check has been added to see if a node is known. If it is not, a thread
in the \ac{AM} is initialized in order to try to authenticate with the other
node, and then goes on to disregard the handling of that packet further, as to
deny it being copied into the routing table by BATMAN.

In schedule the signature offset, or the authentication token as described in
the next section, was added at the end of the extended \ac{OGM}. If there is no
authentication token or signature available, it will add a zero value,
indicating the node is not authenticated with any network.

\section{First Phase}
The first phase of the implementation started with looking through the whole
source code of BATMAN in order to get an understanding of where and how changes
had to be made. The batman class has a nice setup where it goes through many
different checks (if-statements) on a received \ac{OGM} to determine wheter it
should be added to the routing table, if information about the node should be
modified, or if the node should be dropped for one of several reasons.

Throughout all of these checks, there is a label it can jump to if any check
fails, called 'send\_packets'. That is a label in which all logic to the
received oacket is disregarded and the program goes on to schedule its own and and
forward other known nodes' \acp{OGM}. This label is very useful because when
checking if a node is already authenticated, the flow can be altered so that if
a node is unknown the program can jump to this label and disregard the handling
of the received \ac{OGM} after initializing a new authentication thread in
\ac{AM}.
