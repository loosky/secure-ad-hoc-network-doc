\chapter{System Design}
\label{ch:design}
\acresetall

This chapter portrays the design for the proposed implementation of the secure
ad hoc network.

\section{Brief Overview}
The design here is based on a functional pro-active ad hoc routing protocol. The
routing is left to the chosen routing protocol, i.e. \ac{BATMAN}, and the
changes made will not affect how the routing is performed. The system design can
be said to be an extension of the protocol, which requires nodes to be
authenticated and trusted before being allowed into the network. To strengthen
the system each node also has to verify their identity periodically, or they are
dropped from the network.

The network setup start with an out-of-band authentication where a master node,
hereafter \ac{SP}, verifies new nodes. How this is done can be up to the
application, but let us assume that the actors carrying their communication
devices, hereafter nodes, physically meets the \ac{SP} at the scene and
exchange their public key fingerprints.

When a new node is discovered by the \ac{SP} using regular routing announcements
as part of the pro-active routing protocol, the \ac{SP} will invite the new node
to a handshake to establish a trust between the two nodes. The new node will
receive the \ac{SP}'s certificate (proxy certificate), and will after verifying
the fingerprint request a proxy certificate itself. After verifying the node's
fingerprint, the \ac{SP} will issue a proxy certificate with (possibly) the
rights to participate in building the \ac{MANET} by broadcasting its own and
re-broadcasting other trusted nodes' routing announcements.

\subsection{Why use Proxy Certificates?}
\acp{PC}, as described in Section \ref{sect:pc}, is used to delegate
rights on behalf of the issuer. That means that the issuer, i.e. the \ac{SP},
can chose to delegate all or a subset of its rights to the receiver of the
\ac{PC}. This can be very useful in a situation where the nodes themselves are
unable to properly authenticate themselves with their pre-existing \acp{LLPKC}
if the \ac{SP} on the scene has no way to verify their certificates. This can be
true if their certificates are issued by an unknown root certificate (\ac{CA})
or simply if there is no Internet access and the certificate is signed by an
unknown entity (unknown to the \ac{SP}), even if it knows and trusts the root
\ac{CA}.

Also, the \ac{SP} could be interested in giving the node rights the node would
not usually have on this specific scene, depending on the situation. This is
easier to achieve when the \ac{SP} can delegate its own rights.

An important feature of the \acp{PC} is that the \ac{SP} can delegate different
kind of rights, as long as it is a subset of its own rights, to different nodes.
There are countless of different rights that can be useful, given the situation
they are used in, but here is a few possible rights/privileges to give the
reader an understanding of the possibilities they give:

\begin{itemize}
  \item Announce itself - let the \ac{MANET} know of your existence
  \item Re-broadcast other nodes announcements - reshape the network topology
  \item Announce a gateway - give the \ac{MANET} access to another network
  \item Use the gateway - allow you to communicate outside the \ac{MANET}
  \item Send and receive messages with a defined application - full application
  rights
  \item Only receive messages from a defined application - limited application
  rights
\end{itemize}

The different choices are essentially up to the \ac{SP} managing the network.
One can ask why this is necessary, and again it depends on the application. If
you are setting up a \ac{MANET} on the scene of a disaster to assist emergency
personnel, you could have some actors be able to organize the effort by sending
orders/commands to the other actors, while some actors only are allowed to
receive the orders. In this situation it might be of great importance to know
that only verified nodes are able to give commands, but the importance of
getting this information available outweighs the need to verify the nodes/actors
receiving this information.

\subsection{Post-Authentication Operation}
After being issued with a \ac{PC} the newly authenticated node will start
periodically broadcasting a signature message, which is a message with a
pseudo-random string and its signed hash value. The new node will then use a
smaller extract from this signed hash value and append it to its regular routing
announcements from this point on, changing only whenever a new signature
message is constructed and sent. The signature message will be received and
verified by all its direct neighbors, and they will now only allow routing
announcements from that node appended with the correct signature hash value. If
this value is absent or incorrect, the announcement will be dropped and regarded
as a spoofing message.

Whenever a routing announcement is re-broadcasted by another trusted node, that
node will change the signature extract appended to the announcement to its own
signature extract. This means that every node only checks its direct neighbor
for authentication, which is a design choice. This proposal assumes that because
every node is verified by the \ac{SP} in the first place, all nodes in the
network will be able to trust each other, which also means they will trust their
neighbors to properly verify their neighbors again. This technique also helps
prevent a wormhole attack, which will be discussed later.

In order for trusted nodes to learn of newly trusted nodes existence, the
\ac{SP} reularly broadcasts lists containing the id, address and public key of
each trusted node in the network. This needs to be done, because before learning
about a new node the other trusted nodes will not accept any messages from this
node. This means the new node will not be able to exchange its own \ac{PC} with
other nodes directly - only through the \ac{SP}.

The list, hereafter \ac{AL}, also adds some \ac{WOT} like capabilities. The list
is signed by the \ac{SP}, which means the integrity of the list is guaranteed by
the \ac{SP}. This means that if the \ac{SP} should go offline, e.g. it could be
out of range, other trusted nodes in the \ac{MANET} can continue to broadcast
the \ac{SP} on behalf of the \ac{SP} - to ensure all nodes in the network know
each other. This can be especially important when the network grows large and
become fully or partially separated and nodes in one part may not have learnt of
the existence of newly trusted nodes yet. It also applies to trusted nodes who
have been offline while new nodes have been verified, then re-enter the network
while the \ac{SP} is offline. [TODO: kanskje ref til mitt eget essay?]

TODO: kanskje en figur som viser dette og?

\section{Requirements}
Ad hoc networks have some desired characteristics such as quick and inexpensive
setup and being independent of communication infrastructure, but they also
impose great challenges regarding security. The challenges regarding security
can vary depending the purpose and environment of the network which will be
covered in this section.

\subsection{Scenario}
The design and implementation presented in this thesis is mostly based on an
emergency situation scenario, in which communication infrastructure is
unavailable. This thesis will also reflect on some possible requirements given
by a military application.

If there is a major emergency situation such as an earthquake or tsunami, it is
likely that parts or the entire communication infrastructure at the scene
is destroyed or temporarily down. The remaining communication lines will then
probably be congested, such that little communication actually goes through
[TODO: find a ref].

In this situation, it is of great importance that Emergency Personnel, such as
Paramedics, Firemen, Policemen and the Military, are able to communicate
efficiently and therefore independently of the public communication
infrastructure. They need this network in order to manage the the operation, and
therefore availability is probably the most important trait of this network.
Secondly, they should be able to trust the communication on the network - i.e.
messages sent are from whom they claim they to be.

Also, being able to authorize new actors on the scene, such as Red Cross, can be
critical to the operation. These new actors will probably not have the necessary
authentication tokens, i.e. certificates, required by the authentication scheme
in the network.

\subsection{List of Requirements}
TODO: Use Martins requirement paper, add requirements, write about each one, not
only nr1 and 4.

Based on the scenario above these requirements can be extracted and made into
general requirements that needs to be addressed by the system design. The work
presented here is based on several sources, most prevalent being the research
from the OASIS project \cite{oasis_report} \cite{5683058} \cite{nyre2009secure}
and the doctoral project of Eli Winjum carried out at UniK
\cite{ffi_2005_04015}.

\begin{table}[ht!]
	\centering
	\begin{tabular}{ | l | p{11cm} | }
	\hline
	\textbf{Requirement} & \textbf{Requirement Description}\\\hline
		R1 & A node must be authorized in order to get full rights in a
		network \cite{dahill2001secure}, \cite{sanzgiri2002secure}\\\hline
		R2 & A node without a recognized authentication token should be able to become
		authorized if necessary\\ \hline
		R3 & Networks need a master node which handles access control\\\hline
		R4 & Different networks should be able to collaborate
		\cite{ffi_2005_04015}\\\hline
	\end{tabular}
	\caption{Requirements based upon our simplified and general scenario.}
	\label{tab:our_req}
\end{table}

An early study produced security requirements of ad hoc networks demanding
that the routing logic must not be spoofed or altered to produce different
behavior \cite{dahill2001secure}. This means authorization is required (R1)
before someone can partake in routing logic. FFI also requires seamless radio coverage over the
area giving us R4.

\section{Design Overview}
The secure ad hoc network designed here does not change any fundamental workings
of regular ad hoc routing protocols. Assuming all nodes in the network already
have been authenticated, the routing in the network should behave as if there
were no secure extension to the routing protocol.

The proposed design should work with most pro-active ad hoc routing protocols
with limited alterations - but this design is specifically made for the
BATMAN \cite{batman_rfc} routing protocol chosen for its simpler design compared to
e.g. OLSR \cite{clausen2003rfc3626} because it only uses the third layer of the
\ac{OSI} Model \cite{zimmermann1980osi}. How to incorporate this design into
BATMAN will be explained in Chapter \ref{ch:implementation}.

The basic principle of the proposed design is that an authenticated node accepts
other authenticated nodes' routing messages and forwards them as normal, while
discarding routing messages from unauthenticated nodes. One or more nodes in the
network will assume a role as master node(s), with the extra capability of
authorizing new nodes into the network. A special certificate called a \ac{PC}
\cite{rfc3820} will be used for authentication after this authorization has taken
place such that other nodes in the network will be able to authenticate and
accept the new node.


\subsection{Entity Explanation}

Before a simplified example can be given, a few new entities in this design
needs to be explained further. This is the short version, just enough for the
reader to understand the example - the full description of these entities
and why they are necessary will be given later. All of these entities are also
portrayed in Figure \ref{fig:simple_example_entities}. The portrayed entities
will be used as a template for other figures later in this thesis report.

\begin{itemize}
  \item \textbf{\acf{SP}} is responsible for tasks similar to that of a \ac{CA}
  	and has the master role in the network. The \ac{SP} is the entity that
 	 authorizes new nodes and signs their \acp{PC}.
  \item \textbf{\acf{PC0}} is a self-signed \ac{PC} belonging to a
  	\ac{SP}. This \ac{PC} has a certificate depth of 0, thus we refer to it as a
  	\ac{PC0}.
  \item \textbf{\acf{PC1}} is a \ac{PC} signed by a \ac{PC0} (i.e. by the
  private key of the \ac{SP}). All authenticated nodes in one network, has a
  \ac{PC1} signed by a \ac{SP} from that network.
  \item \textbf{\acf{AL}} is a list containing the necessary information about
 	all authorized nodes in the network. The \ac{SP} signs and broadcasts a copy
 	of the \ac{AL} periodically, and all nodes keep a local copy of the \ac{AL}
  	which they use to authenticate other nodes in the network.
  \item \textbf{Authenticated Node} bla bla.
  \item \textbf{Unauthenticated Node} bla bla.
\end{itemize}

\begin{figure}[h]
	\centering
  	\includegraphics{images/simple_example_entities.png}
  	\caption{Different entities in the Simple Example.}
	\label{fig:simple_example_entities}
\end{figure}

\subsection{Simple Example}
TODO: Why proxy certs, and not just regular certs. Explain this very short in
this example.

Two nodes are within transmitting range of each other. One of the
nodes is a \ac{SP} and the other is unauthenticated. The pro-active ad hoc
routing protocol used on both nodes regularly broadcasts routing messages, so
the two nodes learn about each others' existence. Upon reception of a routing
message from the unauthenticated node, the \ac{SP} will invite the node for a
handshake. During this handshake, the unauthenticated node will send a \ac{PC}
request. The \ac{SP} can create and sign a \ac{PC} for this node - i.e. the node
is issued a \ac{PC1}.

Before the \ac{SP} actually signs the \ac{PC} requested from the unauthenticated
node, it needs some verification that the node is an actor that should be
allowed access to the network. The actor will therefore meet the \ac{SP} in the
field and give its public key fingerprint (out-of-band) so the \ac{SP} can
verify the incoming public key in the \ac{PC} request as the actor's request.

Using the public key of the \ac{PC1}, the newly trusted node will now create and 
broadcast a signature, and use an excerpt (offset value) from this signature
in its routing messages. The signature excerpt is just a value used in the
following routing messages, and not the cryptographic signature itself, but
will be used for recognition. This will be described further later in the
chapter. The \ac{SP}, recognizing the signature offset value will rebroadcast
the routing messages so regular ad hoc routing follows.

Also upon handshake completion is the generation an \ac{AL}. The \ac{SP} will
use this list in order to save certain necessary details about the other node,
such as its address, public key, last signature, and more. Also in this list is
the corresponding information about the \ac{SP} itself. When the list is created
or updated it is broadcasted to the network - signed by the \ac{SP} to ensure no
other node can alter the information about trusted nodes in the network.

Whenever a new node is discovered by the \ac{SP} this procedure repeats, and a
new addition is made to the \ac{AL}. Other previously trusted nodes will learn
the identity of new nodes when they receive the updated list from the \ac{SP}.

TODO: add an MSC or figure with the entities to explain the example\ldots

\section{Node States}
This section is devoted to explain the different states a node can be in, and
how it behaves during these different states. These states should be similar for
most pro-active routing protocols as the main thing triggering these phases
is the routing messages (announcements) sent as per normal operation of any
pro-active routing protocol [some def of pro-active routing protocols].

\subsection{Node Discovery}
Upon entering the network area, the node is both unauthenticated and unknown to
the network. The node regularly broadcasts routing messages to be received by
any potential node in the area. Assume all nodes are configured with addresses
on the same subnet so they can receive the sent broadcast (See Limitations in
Section \ref{ip_address_conf}).

Simultaneously, the node also listens to other nodes' routing messages.
Depending on the time interval between the broadcasts and whether the nodes
within each other's transmitting range are asymmetrical, they will discover each
other approximately at the same time.

\begin{figure}[h]
	\centering
  	\includegraphics[width=0.5\textwidth]{images/node_states_discovery.png}
  	\caption{Discovery Phase between a \acf{SP} and an unauthenticated node (A)}
	\label{fig:node_states_discovery}
\end{figure}

Figure \ref{fig:node_states_discovery} illustrate the routing messages
periodically sent by two nodes until they discover each other. One of the nodes
have already assumed the master role and is a \ac{SP} while the other node is
unauthenticated.

The \ac{SP} will have its own self-issued (and signed) \ac{PC0} and its \ac{AL}
has only one entry - its own \ac{PC}. Note that if it had authorized another
node at an earlier point in time (but within the lifetime of the \ac{PC}) that
node's \ac{PC1} would also be represented in the \ac{AL}, even if the node was
outside the network at this point (physically).

The new node does not have any \ac{PC} at this point, unless it has a \ac{PC}
issued within and valid only for another network. This is however not covered
here, and it is assumed the node has no certificate at all. The same goes for
its \ac{AL}, or one can rather say it has an empty \ac{AL} - denoted by the '�'
in the figure.

\subsection{Authentication Handshake}
Once the two nodes have discovered each other, they will enter the
authentication handshake state. Note that while the nodes are executing the
authentication handshake, they will also continue to execute the regular routing
operations because this authentication step will be executed in a separate
thread. This is further elaborated in the next chapter.

The unknown node will wait for a predefined amount of time, for it may not ever
receive a request. The discovery might have been of an authenticated node with a
\ac{PC1} and not a \ac{SP}, which does not have the rights to authenticate new
nodes. It might also have been that of a \ac{SP}, but because of the flaky
nature of ad hoc networks, the two nodes might have become invisible to each
other before the handshake could be initiated.

While the new node is waiting for an invite, the \ac{SP} generate the invite
message which is a message containing the public key, i.e. the \ac{PC0}, and a
list of requirements regarding the allowed cryptographic encryption schemes, key
sizes and so on. The \ac{SP} will then send the invite directly addressed to the
new node. The \ac{SP} will then wait for a certificate request (\ac{PC} Request)
for a predefined time before re-sending the invite. It will do this at a maximum
three(TODO: maybe find a source claiming a number of times is better based on
statistics or whatever..) times before aborting the authentication handshake
with the new node.

Based on the requirements set by the \ac{SP} in the invite message, the new node
will generate an asymmetric key pair and a \ac{PC} request which it will send
back. This request abides by the rules for making a proxy certificate
\cite{rfc3820}, setting the Issuer Name as the Subject Name from the received
\ac{PC0}, the Subject Name as the Issuer Name appended with its own unique
Common Name from a hash of its own public key, and the Serial Number from the
same hash value. Why proxy certificates are used will be discussed later.

As Figure \ref{fig:node_states_handshake} shows, the last message to be sent in
the handshake is the actual signed certificate (\ac{PC1}) from the \ac{SP} to
the new node. At this point the \ac{SP} will add the \ac{PC1} to its local
\ac{AL} (which will be broadcasted later) and the new node will store the
certificate for making signatures later.

\begin{figure}[h]
	\centering
  	\includegraphics[width=0.5\textwidth]{images/node_states_handshake.png}
  	\caption{Handshake between a \acf{SP} and an unauthenticated node (A).}
	\label{fig:node_states_handshake}
\end{figure}

\subsubsection*{Out-Of-Band}
In the section above the different states and messages shared between the two
nodes were discussed. How and and on what grounds the \ac{SP} accepts the
\ac{PC} request from the new node however, was not discussed. There are many
possible authentication schemes, but when it comes to \acp{MANET}, possibly with
no Internet connection, proper authentication becomes a difficult task. In
the discussion in Chapter \ref{ch:discussion} different schemes will be
discussed, but here only one scheme is accounted for.

If you have no pre-shared information between the parties involved in the
network, the simplest way to authenticate a new node is to use an out-of-band
authentication. The implementation of such an authentication scheme will be
discussed in the next chapter, and will only be briefly mentioned here.

New users will have to use share their public key fingerprint with the \ac{SP}
physically, and vice versa. By doing this, the \ac{SP} can store the fingerprint
and use it to check the received public key in the \ac{PC} request for
authenticity in order to make sure the new node is run by the actual person he
met and verified physically. The new node can similarly verify the \ac{SP}.

\subsection{Authorized Operation}
Once a node is authorized and has its own \ac{PC1}, it can send its own routing
announcements, receive announcements, and forward other nodes' announcements.
This means that the node is a fully worthy member of the MANET. However, there
is one thing missing before the node can be verified and verify other nodes
(than the \ac{SP}) - i.e. it has to learn the public key and address of all its
neighbors.

Figure \ref{fig:node_states_authorized} shows the authorized node receiving
an \ac{AL} Update from the \ac{SP}. This message contains the full \ac{AL} list
and lets the newly authorized node learn the public key and address of potential
other nodes in the network. The list is broadcasted by the \ac{SP} periodically
to make sure all nodes in the network know and trust each other.

\begin{figure}[h]
	\centering
  	\includegraphics[width=0.5\textwidth]{images/node_states_authorized.png}
  	\caption{Normal operation between a \acf{SP} and an unauthenticated node
  	(A)including an \ac{AL} Update message.}
	\label{fig:node_states_authorized}
\end{figure}



\subsubsection*{Discovering New Nodes}


\section{Detailed Entity Description}
This section describes the different entities in greater detail. How they are
used and why they are present is some of the questions that will be answered.

\subsection{\acf{PC}}


\subsubsection*{\acf{PC0}}

\subsubsection*{\acf{PC1}}

\subsection{\acf{SP}}
The term \acl{SP} was coined by Dr. Lawrie Brown \cite{lawrie:technotes}.
\acp{SP} are used in place of regular \acp{CA} for \acp{PC}. The \ac{SP}
determines whether a node should be issued a \ac{PC} and if so which policies to
attach. The \ac{SP} drastically distinguishes itself from regular \acp{CA}
because it breaks the hierarchical model usually associated with \acp{CA} when
they are part of a \ac{PKI}.

A \ac{SP} can also be part of a \ac{PKI} (issued a regular public key
certificate), but because of our scenario, it can be difficult or impossible to
verify its regular certificate. Therefore, no distinction is made whether the
\ac{SP} has had a regular ``\ac{CA}-signed'' certificate sign its \ac{PC0} or if
its \ac{PC0} is self-signed.


\subsection{\acf{AL}}


\section{Authentication Module Messages}
This section describes the different messages sent within, or modified by, the
\ac{AM} extension. This includes how they are created, what payload they
contain, if and how the information is secured from malicious actors, and
finally to whom and how often they are sent during normal operation.

\subsection{Authentication Handshake}
The authentication handshake describes which messages are sent while
authenticating a new node to the network. The handshake is triggered by the
event where the \ac{SP} discover a new and unauthenticated neighbor node.

\subsubsection*{Node Discovery}
Node discovery is not actually a part of the handshake itself, but it is the
event that triggers the handshake and is described here to make the context more
clear to the reader.

TODO: Write about the neighbor list before this section\ldots

As described earlier each node stores a local copy of the \ac{AL} and a direct
neighbor node list. Whenever a trusted node receives a routing announcement from
a new neighbor, i.e. a node which is not in the direct neighbor list, the
trusted node needs to check the \ac{AL} to see if the node is a trusted node. If
the node can be found in the \ac{AL} the trusted node will request the signature
of the node and store it alongside the address of the node in its neighbor list.

If the node is an unknown node however, regular trusted nodes, i.e. nodes issued
with a \ac{PC1}, will discard the announcement and go on with its other
operations. \acp{SP} will initiate an authentication handshake in the same
event.

\subsubsection*{Handshake Invite}

\subsubsection*{\acf{PC} Request}

\subsubsection*{\acf{PC} Issue}

\subsubsection*{ACK ?}



\subsection{Signature}

\subsubsection*{Signature Message}

\subsubsection*{Signed Routing Announcements}



\subsection{\acf{AL}}

\subsubsection*{\acf{AL} Update}

\subsubsection*{\acf{AL} Single Row Update}