\chapter{Testing \& Results}
\label{ch:testing_results}
\acresetall

In this chapter two tests are described and their results presented and
discussed. The two tests measure and compare the time performance in two common
stages for both the original implementation of BATMAN, and the extended version
proposed and implemented in this thesis.

\section{Initialization Phase}
The ``initialization phase'' is the setup phase between two or more nodes trying
to create a network. With the original implementation of BATMAN this phase only
consist of two stages; namely discovering a neighbor node, and deciding to add
the node as a direct link, or ``last-hop'' per BATMAN terminology, in its
routing table.

With the proposed design and implementation from this thesis, two more stages
are added. After the discovery, the authentication handshake stage and the
keystream sharing stage are conducted before the last stage where BATMAN adds
the node as a new direct link in its routing table.

\subsection{Hypothesis}
With the modified version of BATMAN proposed in this thesis, one should observe
a small extra delay in the setup of the network, compared to the original
BATMAN protocol. This extra delay should however, not be significantly higher,
i.e. it should be relatively constant and at no time should any linear increase
in delay be observed.

\subsection{Setup}

\begin{figure}[h]
	\centering
	\includegraphics[width=0.5\textwidth]{images/setup_test_1.png}
	\caption{Physical network layout used in test 1. When using the modified version node B acts as the SP of the network}
	\label{fig:setup_test_1}
\end{figure}

Figure \ref{fig:setup_test_1} presents the setup of the test machines used to
conduct this first test. Node A and B are stationary boxes while node C is a
laptop. Their hardware specifications are described in Appendix
\ref{appendix:lab_setup}. The reasoning to use a different hardware for node C
is the need to create distance in the network, and that outside the ethernet
subnet for which the two other nodes were connected to, it would be easier to
use a laptop during setup. In the next test, this laptop is yet again moved
further away.

An important feature to notice about how these nodes were set up is that node A
and C are outside each other transmitting range, meaning they need an
intermediate node to route their packets to and from each other. Node B is
conveniently placed with almost equal distance to each of the two other nodes.

The landscape the nodes are setup in is a typical office landscape, with varying
obstructing materials such as concrete, wood, and glass. A more ideal setup
would naturally be outdoors, as the network is intended for, but with the lack
of mobile nodes and time this became out of the option.

\subsection{Procedure}
In order to get the same behaviour each run for the modified version, each run
had to be run discretly, i.e. after each run the daemon was shut down and
restarted. This way each run will include all four stages explained above:
discovery, authentication handshake, keystream material sharing, and routing
table update. This was also done on the original implementation, even though
there are no authentication steps in between, but in order to have the exact
same procedure each time.

For each run, these steps were followed:
\begin{enumerate}
  \item Start Node A and C
  \item Wait and make sure both nodes are stable
  \item Start Node B
  \item When both node A and C are discovered and added to routing table kill
  all daemons
  \item Record the log from node B
\end{enumerate}

These steps were taken 10 times in order to have a reasonable data set and
average.

\section{Route Convergence}

\subsection{Hypothesis}

\subsection{Setup}

\subsection{Procedure}



\section{Results}

\subsection{Initialization Phase}

\begin{figure}[h]
	\centering
	\subfloat[Original B.A.T.M.A.N.]{\label{fig:test_1_original}\includegraphics[width=0.5\textwidth]{images/test_1_original.png}}
	\subfloat[Modified B.A.T.M.A.N.]{\label{fig:test_1_secure}\includegraphics[width=0.5\textwidth]{images/test_1_secure.png}} 
	\caption{In a network of three nodes, the time spent by the \ac{SP} from its first neighbor discovery and until both neighbors are added to its routing table.}
	\label{fig:results_test_1}
\end{figure}

Note that with this relatively low number of records and high variance, there
are no statistically significant results. But sort of good enough :)

\subsection{Route Convergence}

\begin{figure}[h]
	\centering
	\includegraphics[width=0.8\textwidth]{images/test_2.png}
	\caption{Routing path convergence time observed by a distant source node to another sink node in the network. The source node is only sporadically connected to the network through a mobile intermediate node.}
	\label{fig:results_test_2}
\end{figure}