
% Latex-versjon av ITEM rapportmal.
% Lagd av <lasse.karstensen@gmail.com>, desember 2009.
% Lisens: public domain. 
%
\begin{titlepage}
\begin{center}
\textsc{NORWEGIAN UNIVERSITY OF SCIENCE AND TECHNOLOGY\\
FACULTY OF  INFORMATION TECHNOLOGY, MATHEMATICS AND ELECTRICAL ENGINEERING} \\
\vspace{0.5cm} 
% crop-et fra http://www.ntnu.no/infoavdelingen/selvhjelp/logoer/ntnu/NTNU_engelsk_RGB.png
\includegraphics[scale=0.5]{images/NTNU_logo.png} \\

\vspace{1.0cm}
{\Huge{PROJECT ASSIGNMENT}}
\vspace{1.0cm}

\begin{tabular}{ p{4cm} p{11cm}}

Students:	& Espen Grannes Graarud \\
Master Thesis \\
Title: & Implementing a Secure Ad Hoc Network \\\\
%\vspace{1cm}
Description: & \\
\end{tabular}
{\small{\begin{tabular}{p{15cm}}
\vspace{0.2cm}
 
Ad hoc networks are useful in situations where wireless communication needs to be established quickly, but a secure implementation must also be able to limit access to the network. This is crucial if the implementation is to be used by e.g. emergency responders or in military applications.
\\\\
In order to have a secure and restricted ad hoc network, the network must only accept communication from trusted devices. In addition the network must also know how to process unknown devices trying to gain access.
\\\\ 
B.A.T.M.A.N. is a routing protocol for ad hoc mesh networks and we aim to extend it to support identification and authentication of mobile devices trying to access a restricted ad hoc network.
\\\\

\end{tabular}  }}

\begin{tabular}{ p{4cm} p{11cm}}
Deadline: & 2011-06-29 \\
Submission date: & \today \\
Department: & Department of Telematics \\
Supervisor: & Martin Gilje Jaatun, SINTEF ICT \\
Co-Supervisor: & Dr. Lawrie Brown, UNSW@ADFA SEIT \\\\
\end{tabular}
\vspace{0.5cm}

Trondheim, \today 

\vspace{1cm}
\line(1,0){150} \\
Stig Frode Mj{\o}lsnes, NTNU ITEM

\end{center}
\end{titlepage}

