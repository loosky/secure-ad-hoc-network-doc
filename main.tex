\documentclass[a4paper, twoside, english, 12pt]{book}

\usepackage{graphicx,subfig,url,acronym,fancyhdr,listings,longtable,verbatim,enumerate,courier,caption,tabularx}
\usepackage[latin1]{inputenc}
\usepackage[pdftex,colorlinks]{hyperref}
\usepackage[top=3.0cm, bottom=2.5cm, left=2.5cm, right=2.5cm, bindingoffset=1cm, includefoot]{geometry}

\setcounter{secnumdepth}{3}

\hypersetup{
		bookmarksnumbered,
		linkcolor=black, 		% Color for normal internal links.
		anchorcolor=black,		% Color for anchor text.
		citecolor=black,		% Color for bibliographical citations in text.
		filecolor=magenta, 		% Color for URLs which open local files.
		menucolor=red, 			% Color for Acrobat menu items.
		pagecolor=red, 			% Color for links to other pages
		urlcolor=blue, 			% Color for linked URLs.
}


\setlength{\parindent}{0pt} 
\setlength{\parskip}{2ex}
\addtolength{\headheight}{0.5pt}
\addtolength{\footskip}{0.5pt}


\usepackage{fancyhdr}
\pagestyle{fancy}
\fancyhead{}
\fancyhead[RO]{\bfseries\rightmark}
\fancyhead[LE]{\bfseries\leftmark}
\renewcommand{\headrulewidth}{0.5pt}
\cfoot{\thepage} 


\begin{document}

\definecolor{darkgray}{rgb}{0.95,0.95,0.95}
\lstset{ %
language=c,                % choose the language of the code
basicstyle=\small,       % the size of the fonts that are used for the code
numbers=left,                   % where to put the line-numbers
numberstyle=\footnotesize,      % the size of the fonts that are used for the line-numbers
stepnumber=0,                   % the step between two line-numbers. If it is 1 each line will be numbered
numbersep=5pt,                  % how far the line-numbers are from the code
backgroundcolor=\color{darkgray},  % choose the background color. You must add \usepackage{color}
showspaces=false,               % show spaces adding particular underscores
showstringspaces=false,         % underline spaces within strings
showtabs=false,                 % show tabs within strings adding particular underscores
frame=single,   		% adds a frame around the code
tabsize=2,  		% sets default tabsize to 2 spaces
captionpos=b,   		% sets the caption-position to bottom
breaklines=true,    	% sets automatic line breaking
breakatwhitespace=false,    % sets if automatic breaks should only happen at whitespace
escapeinside={\%*}{*)}          % if you want to add a comment within your code
}

\pagenumbering{roman}
\pagestyle{fancy}
\chapter*{Abstract}
\addcontentsline{toc}{chapter}{Abstract}

In emergency situations and military operations it is useful to be able to quickly establish communication. It is often necessary to accomplish this with minimum pre-existing infrastructure and without centralized administration. In such scenarios it would also be important that the network is secure - not only implying keeping the communication secret, but also be able to restrict access to the network. Wireless ad hoc networks fulfill many of these requirements, but the issue of security and access control still remains a challenging task.
\\\\
The goal of this study can be divided into two parts. The first part was focused on trying to define a system with a proper authentication scheme that does not affect the nature of ad hoc networks. We combined common authentication mechanisms and an ad hoc routing protocol for this purpose. Secondly, the B.A.T.M.A.N. routing protocol was extended to incorporate the very basic functionality of the system design proposed.
\\\\
A small laboratory environment was set up to test the performance of the extended protocol with the intention of proving that our basic functionality did not weaken the unique properties of mobile ad hoc networks. The test results shows that the basic idea of our system design is possible, and that the current implementation should be further extended to fulfill the requirements necessary for a secure ad hoc network.


\pagestyle{empty}
\cleardoublepage
\pagestyle{fancy}

\chapter*{Preface}
\addcontentsline{toc}{chapter}{Preface}

This master thesis report is written by Espen Grannes Graarud and concludes my 5
year master programme in Communication Technology specializing in Information
Security at the Norwegian University of Science and Technology, NTNU.

This thesis is a continuation of my and Anne Gabrielle Bowitz' specialization
Information Security specialization project that was proposed by Dr. Lawrie
Brown of UNSW@ADFA, Australia, and Martin Gilje Jaatun of SINTEF ICT, Norway.

TODO: more

\begin{center}
\vspace{4cm}
\noindent Trondheim, \today
\vspace{2cm}
\\Espen Grannes Graarud
\end{center}


\pagestyle{empty}
\cleardoublepage
\pagestyle{fancy}

\chapter*{Acronyms}
\addcontentsline{toc}{chapter}{Acronyms}

\begin{acronym}

\acro{3G} {3rd Generation Mobile Telecommunications}

%\acro{AC} {Attribute Certificate}

\acro{AL} {Authentication List}

\acro{AM} {Authentication Module}

\acro{BATMAN} {Better Approach To Mesh Ad hoc Networking}

\acro{CA} {Certificate Authority}

\acro{CBC} {Cipher-block Chaining}

%\acro{CRL} {Certificate Revocation List}

%\acro{DHCP} {Dynamic Host Configuration Protocol}

%\acro{DNS} {Domain Name System}


\acro{ECC} {Elliptic-Curve Cryptography}

\acro{EEC} {End-Entity Certificate}

\acro{IV} {Initialization Vector}

\acro{LLPKC} {Long-Lived Public-key Certificates}

\acro{MAC} {Message Authentication Code}

\acro{MANET} {Mobile Ad Hoc Network}

%\acro{MPR} {Multipoint Relay}

\acro{NL} {Neighbor List}

\acro{OASIS} {Open Advanced System for dISaster and emergency management}

\acro{OGM} {Originator Message}

\acro{OLSR} {Optimized Link State Routing}

\acro{OSI} {Open Systems Interconnection}

\acro{PC} {Proxy Certificate}

\acro{PC0} {Proxy Certificate 0}

\acro{PC1} {Proxy Certificate 1}

%\acro{PKC} {Public Key Cryptography}

\acro{PKI} {Public Key Infrastructure}

%\acro{SLC} {Short Lived Certificates}

%\acro{SLCS} {Short Lived Credential Service}

\acro{SP} {Service Proxy}

\acro{SSO} {Single Sign-On}

\acro{TTL} {Time To Live}

%\acro{UAV} {Unmanned Aerial Vehicle}

\acro{Wifi} {Wireless Fidelity (See '802.11' in Definitions)}

\acro{WOT} {Web Of Trust}

\end{acronym}


\pagestyle{empty}
\cleardoublepage
\pagestyle{fancy}

\chapter*{Definitions}
\addcontentsline{toc}{chapter}{Definitions}



\begin{acronym}
%Legg inn i alfabetisk rekkefølge!

%\acro{Ad Hoc Network}
%		A self-organizing network with no form for pre-existing infrastructure or
%		centralized administration.

%\acro{Authenticated List} %egendefinert forkortelse av espen
%	A list containing the public keys, IP, roles, certificate validity period,
%	signature fraction and the timestamp of the last received signature of all
%	authenticated nodes in the network. The list broadcasted by the SP
%	periodically.

\acro{Authentication}
	Say something about authentication

\acro{Authentication Module}
	Addition to the B.A.T.M.A.N. protocol which takes care of cryptographic
	functions and other additions. It also adds fields to the Originator messages
	which can contain a digital signature or signature fractions, and sends other
	messages with nonces, certificates, and ALs.

\acro{Authentication Token}
	Say something about authentication tokes, such as certificates and so on\ldots

\acro{Authorization}
	Say something about authorization

\acro{Congestion}
	Congestion is a state in wich the the amount of traffic on a network surpasses
	the stable amount of traffic the network can handle. I.e. congestion can make
	the network useless if not handled by some control mechanisms.
	
\acro{Certificate Authority}
	Say something about CAs.

%\acro{Convergence Time}
%	The time it takes for the network to get to a stable state with no route
%	flapping after an event that has changed the network topology. E.g. a node has
%	died or moved and made a link inferior to other alternative links.

%\acro{Elliptic-Curve Cryptography}
%	Public key cryptography based on the mathematical properties of elliptic
%	curves.

%\acro{Multicast}
%	In computer networking this refers to the delivery of a packet or message to a
%	group of devices.

\acro{OASIS}
	???

%\acro{Originator}
%	Synonym for a Batman interface which is a network interface utilized by
%	Batman.

%\acro {Originator Message}
%	Batman protocol message advertising the existence of an originator. They are
%	used for link quality and path detection \cite{batman_rfc}. %CP

%\acro{Packet Delivery Ratio}
%	Proportion of delivered packets relative to the amount of packets sent.

\acro{Proxy Certificate} 
	A X.509 certificate signed by a regular X.509 EEC. It is used to assign roles
	to which the recipient can act on behalf of the signee.

\acro{Proxy Certificate 0}
	Say something about PC0
	
\acro{Proxy Certificate 1}
	Say something about PC1
	
	
%\acro{Public Key Infrastructure}
%	Every entities involved with the management (creation, distribution etc.) of
%	public key certificates. Managed by the PKIX working group of IETF.

%\acro{Round Trip Delay}
%	The time it takes from a packet is sent from the sender and the sender
%	receives as acknowledgment packet from the receiver.

%\acro{Route Flapping}
%		Occurs when a node in a network continuously changes preferred route between
%		a source and destination pair creating route instability.


\acro{Service Proxy}
	Say something about SPs.

%\acro{Shortest Path}
%	Minimum number of hops between two communicating nodes.

\acro{Socket}
	Say something about sockets.

\acro{Thread}
	Say something?

%\acro{X.509 Certificates}
%	Standard public key certificate standard managed by the PKIX working group of
%	IETF.

\end{acronym}


\pagestyle{empty}
\cleardoublepage
\pagestyle{fancy}

\setlength{\parskip}{0ex}
\tableofcontents

\pagestyle{empty}
\cleardoublepage
\pagestyle{fancy}

\listoffigures

\pagestyle{empty}
\cleardoublepage
\pagestyle{fancy}

\listoftables
\setlength{\parskip}{2ex}

\pagestyle{empty}
\cleardoublepage
\pagestyle{fancy}

\pagenumbering{arabic}
\chapter{Introduction}
\label{ch:intro}
\acresetall

Today, we have become accustomed with an almost complete presence of digital
networks in our daily lives. Everywhere we go, you can either plug your laptop
into an ethernet slot, connect your laptop to an available wifi hot spot, or
just use your cell phone via 3G mobile data network. However, this is not
universally true throughout the world. Many places are scarcely populated, or
the people living there do not have the resources to deploy such networks.

In emergency and/or military situations, this often applies. Even if it didn't,
the networks may have been put out of operations due to the nature of the
emergency (i.e. tsunami destroying the infrastructure). The military might
also be in an hostile environment where they cannot trust the network in place
altogether.

Emergency search and rescue and military tactical operations can greatly benefit
from the use of digital communication for sharing operation critical
information. If they have no trusted data network available, they should
therefore setup one themselves. A realistic approach would have to be easy and
quick to setup and be self-managing thus requiring minimal maintenance. It
should to an extent always be available to the participants wherever they go,
which calls for using a wireless network. Last but not least, the network needs
to be trusted, i.e. you should trust that the infrastructure is not compromised
and that the communicating parties on the network are who they claim to be.

A \ac{MANET} solves some of these requirements. It does not need an existing
communication infrastructure, it is self-organizing and the network coverage
range can easily be extended by placing intermediate nodes in strategic
locations. The latter requirement however, is a more challenging task in
\acp{MANET}.

With the lack of infrastructure in \acp{MANET} and no guarantees they are
connected to the Internet, establishing trust between the nodes become different
from that on the Internet which can rely on e.g. \acp{PKI}.

In this thesis I will propose and implement a solution suggestion to establish a
trust mechanism, i.e. an authentication scheme, which combines features of a
typical \ac{PKI} with \ac{WOT} ideas [TODO: ref of gpg project maybe?].


\section{Motivation}
The motivation for research in this field is very high because the possible
social benefits are great. For instance, the recent 7.0 magnitude earthquake
that struck Haiti in 2010 [TODO: REF it? News-source or scientific?]
[Wired-article http://www.wired.com/magazine/2010/04/ff_haiti/ ?] showed us how
huge relief efforts easily become very inefficient when huge amounts of
emergency relief personell work at a scene with little or scarce communication
throughout the area. With a trusted communication network like a secure
\ac{MANET} an operation like this could become much more efficient, bringing the
right amount of help to the right places at the right time.

\section{Contributions?}
Specify what I am contributing with?

\section{Objectives}

\section{Limitations}

\section{Method}

\section{Document Structure}
This thesis report is structured as follows:

\textbf{Chapter \ref{ch:background}: Background} aims to give the reader the
necessary insight about the technologies, ideas and theories discussed later in
this thesis.

\textbf{Chapter \ref{ch:design}: System Design} propose an original solution
for an authentication scheme for \acp{MANET}.

\textbf{Chapter \ref{ch:implementation}: Implementation} presents the
implementation of the system design. The implementation is a modification of the
\ac{BATMAN} source code.

\textbf{Chapter \ref{ch:testing}: Testing} devise different tests for checking
the performance of the implementation compared to the original \ac{BATMAN}
implementation, and how well the implementation withstands some known \ac{MANET}
cryptographic attacks.

\textbf{Chapter \ref{ch:results}: Results} presents the results of the tests
from the previous chapter.

\textbf{Chapter \ref{ch:discussion}: Discussion}

\textbf{Chapter \ref{ch:conclusion}: Conclusion}

\textbf{Appendix \ref{appendix:source}: Source Code}

\textbf{Appendix \ref{appendix:irc}: IRC Chat Logs}

\pagestyle{empty}
\cleardoublepage
\pagestyle{fancy}

\chapter{Background}\label{background}

\section{B.A.T.M.A.N.}\label{ad_hoc_network} 

\section{Proxy Certificates}\label{authentication}

\section{Attacks on Mobile Ad Hoc Networks}
\subsection{Wormhole Attack}
\subsection{Blackhole Attack}
\subsection{Greyhole Attack}


\pagestyle{empty}
\cleardoublepage
\pagestyle{fancy}

\chapter{Design}
This chapter portray the design for the proposed implementation of the secure ad
hoc network.

\section{Requirements}
Ad hoc networks have some desired characteristics such as quick and inexpensive
setup and being independent of critical communication infrastructure, but they
also impose great challenges regarding security and more. The challenges
regarding security can vary depending the purpose and environment of the
network.

\subsection{Scenario}
The design and implementation presented in this thesis is mostly based on an
emergency situation scenario, in which communication infrastructure is
unavailable. This thesis will also reflect on some possible requirements given
by a military application.

If there is a major emergency situation such as an earthquake or tsunami, it is
likely that parts or the whole of the communication infrastructure at the scene
is destroyed or temporarily down. The remaining communication lines will then
probably be congested, such that little communication actually goes through.

In this situation, it is of great importance that Emergency Personell, such as
Paramedics, Firemen, Policemen and the Military, are able to communicate
efficiently and therefore independently of the public communication
infrastructure.

They need this network in order to manage the the operation, and therefore
availability is probably the highest regarded trait of this network. Secondly,
they should be able to trust that the communication on the network is to be
trusted, i.e. messages sent are from who they claim they to be.

Also, being able to authorize new actors on the scene, such as Red Cross, can be
critical to the operation. These new actors will probably not have the necessary
authentication tokens, i.e. certificates, required by the authentication scheme
in the network.

\subsection{List of Requirements}
Based on the scenario above these requirements can be extracted and made into
general requirements that needs to be addressed by the system designm.

\begin{table}[ht!]
	\centering
	\begin{tabular}{ | l | p{11cm} | }
	\hline
	\textbf{Requirement} & \textbf{Requirement Description}\\\hline
		R1 & A node must be authorized in order to get full rights in a
		network\\\hline
		R2 & A node without a recognized authentication token should be able to be
		authorized if necessary\\ \hline
		R3 & Networks need a master node which handles access control\\\hline
		R4 & Different networks should be able to collaborate\\\hline
	\end{tabular}
	\caption{Requirements based upon our simplified and general scenario.}
	\label{tab:our_req}
\end{table}



\section{Design Overview}
The secure extension of BATMANasdasdasdasd

\pagestyle{empty}
\cleardoublepage
\pagestyle{fancy}

\chapter{Implementation}
\label{ch:implementation}
\acresetall

This chapter goes into depth on how the design from the previous chapter is
implemented into the BATMAN protocol. Some code snippets will be shown in the
following sections when applicable, and the full source code can be found in
Appendix \ref{appendix:source}.

%\begin{figure}[h]
%	\centering
%	\includegraphics[totalheight=1\textheight]{images/batman_if_statements.png}
%	\caption{IF-Statements in the Batman Class}
%	\label{fig:batman_if_statements}
%\end{figure}

\section{Authentication Module}
Almost all functionality added to BATMAN is within the borders of a separate
class called \ac{AM}. The first thing to notice about the \ac{AM} is that it
runs in its own thread and sockets, so all authentication mechanisms run
concurrent to regular BATMAN routing operations. This separation was necessary
in order to have BATMAN behave normally during e.g. the authentication of a
node, so a large network should not suffer if an important node (centrally
located) is 'hung up' in e.g. authenticating another node.

\subsection{AM Thread}
In the setup phase in the original batman class, an \ac{AM} initiation function
called \texttt{am\_thread\_init} from the \ac{AM} class is called. This function
takes the network interface name and its corresponding IP address and broadcast
address as input. These values are then stored locally in in the AM class for
socket setup. For socket setup see Section \ref{subsect:am_socks}.

The function then goes on to create a new thread for the AM module which is the
main thread taking care of most of the additions in this implementation. The
thread first sets up two sockets for sending and receiving AM messages such as
handshakes and keystream-materials.

Next it generates a highly random (high entropy) master key using the OpenSSL
function \texttt{RAND\_bytes}, which has been properly seeded - see Section
\ref{subsect:posix.c}. This and an IV generated with OpenSSL's
\texttt{RAND\_pseudo\_bytes} is then used to generate a master key encryption
context for AES encryption, before deleting the master key. With the encryption
context the necessary internal memory used by OpenSSL for encrypting with this
key is stored, and therefore the key and IV is of no more use by themselves -
making it a sound security choice to delete the key (and IV) entirely.

The next important action is to generate either a \ac{PC0} or a \ac{PC} request
depending on whether you are a \ac{SP} or just a regular node trying to
authenticate with the network.

After these steps the ``initiation phase'' of the AM thread is complete, and the
rest of the code runs in a loop until the BATMAN daemon is terminated.

\subsection{AM Sockets}\label{subsect:am_socks}
Two sockets are used for the AM module, one for sending and one for receiving AM
messages. Both sockets are bound to the interface device given by the AM
initiation function. The receive socket is then bound to a designated port
64305, regular BATMAN runs on port 4305, and the send socket is explicitly
allowed to send to broadcast addresses.

Sockets needs to be explicitly set to be allowed to send to broadcast addresses
in UNIX systems, as a protection mechanism.

The following shows a code snippet of how the sockets are setup:
\begin{lstlisting}[frame=tb]
int32_t *recvsock, *sendsock;
addrinfo hints, *res;

/* Set family information */
memset(&hints, 0, sizeof hints);
hints.ai_family = AF_INET;
hints.ai_socktype = SOCK_DGRAM;
hints.ai_flags = AI_PASSIVE;
hints.ai_protocol = IPPROTO_UDP;

/* Puts the port-info inside a addrinfo data structure */
getaddrinfo(NULL, port, &hints, &res);

/* Assign file descriptor for sockets */
*recvsock = socket(PF_INET, SOCK_DGRAM, 0)
*sendsock = socket(PF_INET, SOCK_DGRAM, 0)

/* Binds the sockets to the network interface */
setsockopt(*recvsock, SOL_SOCKET, SO_BINDTODEVICE, interface, strlen(interface) + 1)
setsockopt(*sendsock, SOL_SOCKET, SO_BINDTODEVICE, interface, strlen(interface) + 1)

/* Binds receive socket to the port (rest of the address is empty/null) */
bind(*recvsock, res->ai_addr, res->ai_addrlen);

/* Allow the send socket to send broadcast messages */
int broadcast_val = 1;
setsockopt(*sendsock, SOL_SOCKET, SO_BROADCAST, &broadcast_val, sizeof int)

/* Set the send socket to non-blocking */
fcntl(*sendsock, F_SETFL, O_NONBLOCK);

\end{lstlisting}

There are several practical reasons to choose UDP sockets over TCP sockets for
this implementation. First and foremost, this system sends authentication
handshake messages and keystream-material messages to nodes which has no route
in the routing tables. If a connection-oriented protocol would try this the
messages would be blocked on the kernel level and not sent. With an
connectionless protocol like UDP no mechanisms will block this message being
sent, it will just send the message not bothering whether the message is ever
received by the recipient.

Second, TCP was created with wired networks in mind, observing much less packet
loss. In \acp{MANET} the packet losses are much higher than in wired and fixed
infrastructure networks, and as nodes move around direct paths between nodes
change much more frequently. TCP is not suitable for such environments because
it will lead to a huge amount of re-sending of packets to ``non-existant''
neighbors and much memory wasted in connection states being kept for dead links.

\subsection{Main Operation of the AM Thread}
Most of the interesting operation in the AM class happens within a single loop
running throughout the lifetime of the BATMAN daemon. The program flow
throughout this loop is shown in Figure \ref{fig:am_main_loop}.

\begin{figure}[hb!]
	\centering
	\includegraphics[width=\textwidth]{images/am_main_loop.png}
	\caption{Main program flow in AM class.}
	\label{fig:am_main_loop}
\end{figure}

For clarity the figure leaves out certain details, but the most important
features are depicted. The handle incoming messages are shown as a sub-process
in the figure.  This is not true, but depicting all possible messages would not
fit the figure, and therefore left out altogether. However, each message are
described in the following sections.

The ``New Neighbor'' is one of the elements in the AM class that must be
triggered by the batman class. If the batman thread tells the AM thread a new
neighbor is discovered and the AM thread is in a ready state to handle new
neighbors the appropriate action is taken, whether the neighbor has been
authenticated with the network or not. Not shown in the figure is how a regular
authenticated node will act if a new neighbor which is not authenticated with
the network is handled, but this is taken care of in the batman class and not
here.

The two last parts should be self-explanatory, they simply check the current
time and then compares this against time values set on the nodes own current
keystream to check if its old, or if any keystreams from other neighbors in the
\ac{NL} are old and if so takes the appropriate action. Also checked is if a
nodes own keystream is getting exhausted, in which also the appropriate action,
namely creating and sharing a new one with each neighbor in the \ac{NL}.


\section{Proxy Certificates}
The \acp{PC} in this implementation are containers for short lived 1024 bits RSA
public keys used in a single session only. Most of the design choices were taken
into the implementation, but some however, were left due to time constraint.

One such was that the original proxy certificate X.509v3 extension, introduced
in RFC 3820 called ``proxyCertInfoExtension'' was not used to carry the policies as
intented. Most of the OpenSSL documentation is not released to the general
public for free, and this X.509v3 extension was no exception. The only examples
found, amongst the original proxy certificate implementations from the Globus
Project\footnote{Globus Project:
\url{http://www.globus.org/toolkit/downloads/}}, but using the exact same setup
did not work in my implementation. After some investigation it seems no open
source code projects using proxy certificates have been publiished for years,
giving me the idea that maybe the OpenSSL specifications have changed during
the last version updates and that proxy certificates have to be implemented
differently. The author have asked for an answer on this subject on both emails
to the developers of OpenSSL, the OpenSSL's mailinglists, and on an
OpenSSL ``IRC'' channel\footnote{OpenSSL IRC Channel: \#openssl on
\url{irc://irc.freenode.net}}.

As a replacement for this extension, a commonly used free-text extension called
\texttt{netscape\_comment} has been used and the policy has been written in
cleartext inside this comment. Because the design proposed used the
\texttt{id-ppl-anyLanguage} and allows the application to decide the language
the policy is written in, this should make no practical difference, other than
not being a strictly RFC 3820 proxyCertInfoExtension.

\subsection{Generating PC Requests}
\subsection{Generating PCs}
\subsection{Verifying PCs}

\section{Authentication List}

\section{Neighbor List}

\section{Keystream}
\subsection{Generating Keystreams}
\subsection{Using One-Time Passwords From Keystream}
\subsection{Verifying One-Time Passwords}

\section{Changes to the BATMAN Protocol}

\begin{figure}[h]
	\centering
  	\includegraphics{images/extended_ogm.png}
  	\caption{Modified routing announcement (Originator Message - OGM) for the modified BATMAN protocol.}
	\label{fig:extended_ogm}
\end{figure}

\subsection{POSIX.C}\label{subsect:posix.c}
\subsection{BATMAN.C}
\subsection{SCHEDULE.C}

\pagestyle{empty}
\cleardoublepage
\pagestyle{fancy}

\chapter{Testing \& Results}
\label{ch:testing_results}
\acresetall

In this chapter two tests are described and their results presented and
discussed. The two tests measure and compare the time performance in two common
stages for both the original implementation of BATMAN, and the extended version
proposed and implemented in this thesis.

\section{Initialization Phase}
The ``initialization phase'' is the setup phase between two or more nodes trying
to create a network. With the original implementation of BATMAN this phase only
consist of two stages; namely discovering a neighbor node, and deciding to add
the node as a direct link, or ``last-hop'' per BATMAN terminology, in its
routing table.

With the proposed design and implementation from this thesis, two more stages
are added. After the discovery, the authentication handshake stage and the
keystream sharing stage are conducted before the last stage where BATMAN adds
the node as a new direct link in its routing table.

\subsection{Hypothesis}
With the modified version of BATMAN proposed in this thesis, one should observe
a small extra delay in the setup of the network, compared to the original
BATMAN protocol. This extra delay should however, not be significantly higher,
i.e. it should be relatively constant and at no time should any linear increase
in delay be observed.

\subsection{Setup}

\begin{figure}[h]
	\centering
	\includegraphics[width=0.5\textwidth]{images/setup_test_1.png}
	\caption{Physical network layout used in test 1. When using the modified version node B acts as the SP of the network}
	\label{fig:setup_test_1}
\end{figure}

Figure \ref{fig:setup_test_1} presents the setup of the test machines used to
conduct this first test. Node A and B are stationary boxes while node C is a
laptop. Their hardware specifications are described in Appendix
\ref{appendix:lab_setup}. The reasoning to use a different hardware for node C
is the need to create distance in the network, and that outside the ethernet
subnet for which the two other nodes were connected to, it would be easier to
use a laptop during setup. In the next test, this laptop is yet again moved
further away.

An important feature to notice about how these nodes were set up is that node A
and C are outside each other transmitting range, meaning they need an
intermediate node to route their packets to and from each other. Node B is
conveniently placed with almost equal distance to each of the two other nodes.

The landscape the nodes are setup in is a typical office landscape, with varying
obstructing materials such as concrete, wood, and glass. A more ideal setup
would naturally be outdoors, as the network is intended for, but with the lack
of mobile nodes and time this became out of the option.

\subsection{Procedure}
In order to get the same behaviour each run for the modified version, each run
had to be run discretly, i.e. after each run the daemon was shut down and
restarted. This way each run will include all four stages explained above:
discovery, authentication handshake, keystream material sharing, and routing
table update. This was also done on the original implementation, even though
there are no authentication steps in between, but in order to have the exact
same procedure each time.

For each run, these steps were followed:
\begin{enumerate}
  \item Start Node A and C
  \item Wait and make sure both nodes are stable
  \item Start Node B
  \item When both node A and C are discovered and added to routing table kill
  all daemons
  \item Record the log from node B
\end{enumerate}

These steps were taken 10 times in order to have a reasonable data set and
average.

\section{Route Convergence}

\subsection{Hypothesis}

\subsection{Setup}

\subsection{Procedure}



\section{Results}

\subsection{Initialization Phase}

\begin{figure}[h]
	\centering
	\subfloat[Original B.A.T.M.A.N.]{\label{fig:test_1_original}\includegraphics[width=0.5\textwidth]{images/test_1_original.png}}
	\subfloat[Modified B.A.T.M.A.N.]{\label{fig:test_1_secure}\includegraphics[width=0.5\textwidth]{images/test_1_secure.png}} 
	\caption{In a network of three nodes, the time spent by the \ac{SP} from its first neighbor discovery and until both neighbors are added to its routing table.}
	\label{fig:results_test_1}
\end{figure}

Note that with this relatively low number of records and high variance, there
are no statistically significant results. But sort of good enough :)

\subsection{Route Convergence}

\begin{figure}[h]
	\centering
	\includegraphics[width=0.8\textwidth]{images/test_2.png}
	\caption{Routing path convergence time observed by a distant source node to another sink node in the network. The source node is only sporadically connected to the network through a mobile intermediate node.}
	\label{fig:results_test_2}
\end{figure}

\pagestyle{empty}
\cleardoublepage
\pagestyle{fancy}

\chapter{Discussion}
\label{ch:discussion}
\acresetall

\section{Modified Routing Announcements Vulnerability}
The design was changed a bit during implementation, and with limited time some
parts did not make it to the implementation These things needs to be addressed
before the system can be used in a real world scenario. There are two attacks
that the system are still vulnerable to, namely the \emph{wormhole attack} and
the \emph{suppress replay attack}.

Because the routing announcements are unencrypted and the one-time-passwords
are just appended after the messages, these packets can be altered using the
following two attacks. In this section the two attacks are first explained, and
then the section goes on to describe possible solutions for both attacks.

\subsection{Wormhole Attack}
In a wormhole attack, see Figure \ref{fig:wormhole_attack}, an attacker does not
need to know the keystream of a node in order to send that node's routing
announcements. If you look at the figure from the background chapter you have a
network of trusted nodes in green, and two malicious nodes in red. 

Assume the two attackers, M1 and M2, want to disrupt the network topology by
having node B believe node A (and vice versa) is a direct neighbor. Assuming the
out-of-band wormhole in the figure is a faster route between node A and B than
the ``real'' route through the network, M1 can simply forward the announcements
from A to M2. Then M2 does the same, except he also needs to spoof his network
and link layer addresses. When B receives the routing announcement from M2, he
believes it is from A and that A is a new direct neighbor.

Now B will ask for A's keystream-material, and M2 and M1 will forward this
request to A, which will dutifully reply with his keystream-material. Note that
M1 needs to spoof B's addresses here, as well as forwarding B's routing
announcements. When B receives the keystream through the wormhole, he is now
able to verify the routing announcements also sent through the wormhole, and
soon will this route take priority in his routing table as a strong direct link
instead of going through other nodes in the network to reach A.

TODO: MSC FIGURE VISER DETTE

This attack only distorts the network topology, but notice that because the
routing announcements' integrity are not protected, and the content is not
encrypted, the attackers can possibly do greater damage by altering the content
of the message. By altering the content, they could possibly generate fake
re-broadcast announcements announcing untrusted nodes, possibly connecting a
whole network of malicious nodes to this trusted network. This alteration is
not actually a wormhole attack, but it is fair to say that the wormhole attack
opens up to a variety of other attacks, such as alteration of packet, dropping
of packets etc.

\subsection{Suppress Replay Attack}
Another attack which is possible due to the fact that the one time passwords are
not ``connected'' to the routing announcements in any way is the suppress replay
attack. Suppose an attacker is able to jam the signals for a very short time,
only enough to distort the main payload of the routing announcement, while the
one time password and its sequence number are kept intact.

The trusted receiver of this distorted packet will ignore it, because he will
not understand the meaning of the packet. The attacker on the other hand, knows
that the following un-corrupt data is a valid one time password and its sequence
number from the sender's keystream.

Because the original recipient of this packet did not understand the destroyed
packet, he will not know that this \ac{OTP} is already used, and if the attacker
wishes to create a false routing announcement he can now do this and append the
\ac{OTP}, which will be accepted by the original recipient. Note that also here
the attacker needs to spoof the addresses of the original sender of the
announcement which he partly jammed.

\subsection{Possible Solution to the Suppress Replay Attack}
A possible solution to the suppress replay attack, is to ensure message
integrity in addition to the authentication. The basic idea would be to create
and append a message digest to the announcement while using the keystream to
encrypt the message, e.g. by using a strong stream cipher.

There is one great pitfall to this idea, however. Because the routing
announcements are often identical to each other, the attacker might be able to
do a known plaintext attack, being able to first recreate the message digest,
and depending on the stream cipher just XOR the announcement and digest with the
encrypted message, revealing the keystream used by the stream cipher. If this
message was in any way blocked from reaching the intended recipient, the packet
could be changed and XOR'ed with the retrieved keystream and sent to the
recipient, which would trust this packet.

One possible way to avoid this problem would be to add some randomness to the
``known part'', namely the original announcement. If the announcement is e.g.
appended with a part of the senders keystream (\ac{OTP}) before creating the
message digest, the attacker would not be able to recreate the full plaintext
(including digest) and would therefore not be able to find the keystream used by
the stream cipher.

The reader might raise another question, regarding the length of this random
part added to the announcement before creating the digest. One major factor
here is time, and time it takes for an adversary to find the random part (collision).
The time is however very restricted. The sliding window used to avoid replay
attacks is only 64 bits long, and in the worst case (best case for adversary)
there are only two legitimate nodes in the network. If this is the case, each
node will send two announcements every second (BATMAN protocol) meaning the
window of opportunity for the attacker is only 32 seconds. For each extra
legitimate node in the network this window drastically closes. The length of the
random part used for the digest should therefore be strong enough withstand 32
seconds of brute-force guessing attack.

Note that it might not be wise to send the whole message digest because the
announcement sizes would become very large. This might be truncated as well, but
when designing this the time needed to find a collision for a truncated digest
needs to be addressed as well.

\subsection{Possible Solution to the Wormhole Attack}
This attack vector is very difficult to to protect against. With the solution
above, many attacks dependent on the wormhole attack are thwarted because of
packet secrecy, integrity and authentication. However, the solution does not
hinder an attacker from replaying the exact same packets through a wormhole, in
order to alter the network topology. There has been a lot of research to find a
good solution to this attack, but most solutions are aimed towards stationary
networks and not \acp{MANET} \cite{raoteapproaches}.

One possible solution as pointed to by the article above is the use of
``location aware guard node'' and graph theory \cite{poovendran2007graph}
\cite{lazos2005preventing} to detect wormholes. The idea is that if you have
special nodes spread out in the field at fixed points, where none of these nodes
are within each other's transmitting range, one should never be a neighbor of
more than one node at a time. If you receive direct packets from two nodes
within a very short time frame, there might be indications that there's a
wormhole in place replaying one of the special nodes' announcements.

\section{Key Usage}
In the fields of information security and cryptography it is not considered a
good practice to use the same keys for different type of tasks such as
encryption for confidentiality and signing for integrity and authenticity. It is
argued that using only one key for different purposes brings a single point of
failure to the design. While this is true, sometimes the benefits of having
single key for these purposes are greater than the risk. In the proposed design
the same public key pair is used to digitally sign keystream-material messages
as well as when encrypting (with the recipient's public key) the symmetric keys
in the same messages. In addition, the \acp{SP} use the same key to sign other
nodes' \acp{PC}.

The benefits of only having a singe public key pair for each node in the network
is simplicity in the design. When it comes to the possible vulnerability if the
one key is compromised, the lifetime of each key which is bound by the
relatively short lifetime of \acp{PC} makes the window of opportunity for misuse
of this key short and damage low.

\section{Future Work - Extending the System Design}
The specialization project \cite{bowitz_graarud} that preceded this thesis
mentioned other important features the secure ad hoc network implementation
should have. The things mentioned in this thesis' system design are actually
implemented, but there are still much more that should be added if this system
should be used in real life emergency situations.

In this section some important features which should be added, or at least
studied, are mentioned. They do add more complexity to the system, so it is
probably better to do a complete security and performance analysis on this
thesis' proposed design before adding these features.

\subsection{Initial Authentication with Long-Lived Public Key Certificates}
One limiting factor in the system design is the need for an out-of-band
initial authentication. With this limitation, every actor in the emergency
scenario needs to manually verify his or her identity to the network management
handled by the \ac{SP}. With many actors, which would be typical in a large
emergency situation like a natural disaster, this process might take up valuable
time from the actual emergency work - which contradicts the whole meaning of
setting up the \ac{MANET} at the scene in the first place.

Now, if an actor has possession of a regular \ac{LLPKC} which the network
\ac{SP} is able to verify, this should be allowed without the need of an
out-of-band authentication. After the \ac{SP} has verified the certificate, he
can now issue the the actor a proxy certificate, signed by him so that all nodes
in the network are able to verify the new node's identity.

The question of whether the \ac{SP} is able to verify the \ac{LLPKC} is not
necessarily easy to answer. If the \ac{SP} trusts the identity and knows the
public key of the issuer of the actor's certificate, he is able to verify that
this actor was trusted with this identity (and rights) at some time. However, it
does not mean this is true anymore - the certificate might have been revoked.

In the absence of Internet access, \acp{CRL} might not be available to the
\ac{SP}. If this is the case, an evaluation of what to do with the actor has to
be done. Ideas that comes to mind might be to issue a very short lived \ac{PC},
maybe with limited rights, to the actor's node so that it can start working now,
but has to re-authenticate later when \acp{CRL} has been brought to the scene
either out-of-band or by setting up Internet access. If no \ac{CRL} is ever
brought to the SPs attention, he might want to require an out-of-band
authentication later.

Either way, as this does not have to happen in the same out-of-band fashion as
the proposed design in this thesis requires, one could also allow the
authentication to happen even if the actor is not a direct neighbor of the
\ac{SP}, but is connected through other nodes in the network. While in this
implementation regular trusted nodes drop the announcements from new
unauthenticated nodes, they could rather tunnel the announcement directly to
the \ac{SP} and have the authentication handshake go through them.

\subsection{Network Merging}
For this implementation to be really useful for emergency services, which will
consist of different types of actors and organizations, the implementation
should support merging of multiple ad hoc networks. For example, if paramedics
set up their own network, and firefighters arrive with their own network,
co-operation between these networks should be supported.

Figure \ref{fig:scenario_networks} (from the specialization
project \cite{bowitz_graarud}) shows a possible scenario when a large emergency
situation such as a natural disaster has happened. The figure portrays 4
organizational actors, i.e. police, firefighters, paramedics, and the military,
which all run their own networks. Some of the networks co-operate through the
use of gateways, such as between the police and the other actors, and the
paramedics and firefighters co-operate by merging their networks.

\begin{figure}[h]
	\centering
  	\includegraphics[width=\textwidth]{images/scenario.png}
  	\caption{Scene from scenario with actors from multiple organizations.}
	\label{fig:scenario_networks}
\end{figure}

Choosing whether to merge two networks or not is mainly a security issue. Some
actors might have higher security concerns than others, e.g. military units does
not trust the other actors to merge with their networks, but they might still want
to communicate with them. This can for example be handled by having certain
gateway nodes being allowed to communicate with nodes outside the network in a
controlled manner.

\subsubsection*{Full Merging}
A \emph{full merging} means that two or more separately built and maintained
networks completely merge so that each node in each of the networks receive
all the routing announcements being sent within the networks. What this
basically means is that the different networks become a single network.

The transition to from multiple to one network is not easy. Up to this point
the assumption has been that there is only one management node called \ac{SP}
in the network. This will now change. With multiple networks merging into one
there will be multiple \acp{SP} managing the access control of the network. In
the next section having multiple \acp{SP} is discussed in greater detail.

There is one more major factor which needs to be addressed with full merging of
networks, namely how access control is defined. One now needs to assume that the
\acp{SP} either use the same policy languages to define what rights a node have
been attributed with, or that they have a way of translating these policy
languages so that they can be uniformly understood. For instance, in Section
\ref{subsect:detailed_pc_descr} three special attributes and their values were
used as an example of how a policy might look like:

\begin{itemize}
  \item Role - Node's role in the network
  \item Routing - Whether the node can partake in the routing
  \item Application - Whether the node has access to the application layer
\end{itemize}

Now, these attributes would seem to make sense in most applications of this
system design, but some implementations might even have more attributes. Their
values might also be different, because different networks might have different
needs, for instance the 'role' attribute might have different values depending
on what kind of elements are present in the network. For instance, one network
might have the ``location aware guard nodes'' mentioned earlier in the network
for wormhole protection, while other network do not. With two such different
networks, full merging might not be feasible, but with smaller differences the
obstacles might be manageable.

\subsubsection*{Limited Merging}
\emph{Limited merging} is a completely different technique than the one
described above, and probably easier to implement. With limited merging of two
or more networks, the networks stay autonomous with their own (possible)
hierarchy, set of rules, management nodes, and so on. What constitutes the
merging is that at least one node in each network is set up with gateway
capabilities, which can broadcast the existence of another network to the nodes
inside its own network.

Rules on the application layer must necessarily still be looked at, as they
should work over the different networks, but routing rules and other access
controls can stay hidden between the networks.

\subsubsection*{Internet Gateway}
Similar to limited merging, a node in a network may be given gateway capabilities
to announce Internet access to the network. Because all nodes belong to the same
\ac{MANET} and the management is consistent within this one network, this
capability should be relatively easy to implement. The only thing the reader
should notice is that the policy field within the proxy certificates of the
nodes can now be used to declare whether the node should have Internet access or
not by adding an \emph{Internet attribute} to the policy.

\subsection{Multiple Service Proxies}
Whenever full merging of two secure and managed MANETs happen as described above
two \emph{or more} (!) SPs will end up in the same network (the emphasis on more
is used because one network might have more than one SP). This would be a
challenging task to overcome, for these networks are inherently flat, or
non-hierarchical. Decision making in the network after this point would be
difficult if one network (or specifically one SP) sees itself as more important
than the other.

It is probably better to say that for two networks to fully merge, they should
see themselves as equal counterpart where no network has more rights than the
other, and therefore the \acp{SP} have the same rights as each other. This means
that nodes issued by one \ac{SP}, or said differently one SP's children, are
equally trusted by all other nodes in the network, depending on the restrictions
set in their \acp{PC} of course.

If all the \acp{SP} in the network are considered ``equal'', then they are all
equally trusted to issue new nodes and introduce them to the network. The major
requirement missing for this scheme to happen is that all nodes in the network
now needs to know the public keys for all SPs.

It is likely that not all nodes in the network will ever become a direct
neighbor and therefore not receive the other SPs' \acp{PC0}, but they will meet
nodes signed by those SPs. A way around this problem is to have each \ac{SP} to
once in a while broadcast a digitally signed list to all its children nodes
containing the identities (unique subject names) and public keys for all other
trusted \acp{SP} in the network. Because this list is signed by a known
\ac{SP}, you are able to verify the authenticity of this list and therefore
implicitly verify and trust all nodes issued with PCs from the other SPs.

\section{Experience with OpenSSL}
The experience with OpenSSL has been a challenging one. There is little
documentation available, and only on some segments of the library. Many
functions being used in the implementation of this thesis are undocumented, or
at least not in a freely available documentation. Not to be confused, you can
find most OpenSSL functions in their documentation, but most of those functions
are not explained, many do not explain output and input to functions, and many
functions have unclear names making it a game of ``guessing'' which functions to
use.

Most of the implementation has been understood by looking at a few available
examples online at different sites, some demos within the OpenSSL libraries,
and by asking the online community. For reference, I would personally recommend the
community at \emph{Stack Overflow}\footnote{\url{http://stackoverflow.com}}
rather than the OpenSSL mailinglists\footnote{\url{openssl-dev@openssl.org}}
\footnote{\url{openssl-users@openssl.org}} which have not answered any
questions I raised.

When I've had enough time I've documented the OpenSSL functions used in my
implementation, which can hopefully help someone planning to further implement
on this design, or other designs with similar functionality.

TODO: mention why ECC not used, problems with ECIES algorithm

\pagestyle{empty}
\cleardoublepage
\pagestyle{fancy}

\chapter{Conclusion}
\label{conclusion}

In this project we have proposed a solution where we combine common security measures to provide a restricted ad hoc network. Public key certificates are used as a means for access control to a network and they are managed by semi-central nodes. The proposal solves the task of issuing certificates by giving trusted nodes rights on behalf of the semi-central nodes to include new nodes to the network, and to be able to verify other trusted nodes without the presence of the semi-central nodes. We accomplish this without significantly affecting the security or the performance of the network.
\\\\
This project started with a comprehensive study of ad hoc networks and their characteristics that was important to consider when trying to find a solution. We also researched the necessary background information about the tools we planned to use - evaluating some routing protocols and different cryptographic schemes that seemed fit in an ad hoc environment.
\\\\
Along with this research it was important to consider the environments and scenarios the ad hoc network was to be used as they would probably influence the network behavior.
\\\\
We used an ad hoc routing protocol called BATMAN as the base for our entire system. The protocol is simple and robust with limited complex functionality compared to many of its alternatives - making it easier to modify for security purposes and to verify its security. The proposed solution adds several new entities to the original BATMAN protocol, which is used in order to achieve a scalable authentication scheme.
\\\\
The proposed system incorporates access control without being dependent on a central authority, rather some entity taking charge at the place and time of the network initialization. We also focused on having the network being able to continue its operation and partially allowing new entities and fully verifying already trusted entities even if the entity with the master role is unavailable.  Even better, several entities can mirror the master role making the authentication scheme very scalable.
\\\\
For full authentication our system either requires dependence on a central authority or an out-of-band authentication taking place on site, both of which might be impractical for ad hoc networks. However, we argue that this is required in order to have proper authentication. When establishing a network this might be optional for the users, and will most likely be depending on the scenario. Either way, our system supports both settings.
\\\\
A simple implementation was developed as a proof of concept and tested to indicate the impact the additional functionality has on the network. It is in no way a secure implementation, but we have modified the protocol in such a way that it could eventually be extended to incorporate the complete system.
\\\\
The results from the two first performance tests seemed to prove our hypothesis that our handshake process adds significantly time overhead during establishment. Our choice of using a random master node allocation seems to be a poor choice and separating the pre-authentication steps from routing messages is definitely recommended - both have been taken into consideration in our proposed system design.
\\\\
We performed two additional tests with far more promising results. We assumed that our implementation would not influence regular BATMAN operation, and tested this assumption by checking route convergence times for already authenticated entities and compared the results with the original BATMAN protocol. We did not discover any discrepancies between the two setups which implies that the development of our implementation should be continued.
\\\\
Our system design accomplishes to provide a proposal for a restricted ad hoc network that takes into account the challenges of their unique nature. Our implementation shows that the BATMAN protocol is a good choice to utilize when trying to implement our solution and that the work done in this project is a step in the right direction.

\section{Further Work}\label{further_work}
This section suggests some topics and areas that would be interesting to investigate further. These topics are omitted due to time constraints, but they important and needs to be looked at later on.
\\\\
A simulation of our system design is recommended in order to investigate how the networks would behave in larger scale and rapidly changing environments. Based on the results from the simulations, changes could be made to the design. It would also be desirable to continue the implementation of our solution and to do some real life tests to see how the system would behave.

\subsection{In-depth Security Analysis}
The design should be tested with some gray box security analysis, i.e. analyzing the specifics of our implementation, not the cryptographic functions that will be used. In addition, a cryptanalysis of the actual implementation would also possibly strengthen the protocols claims.


\pagestyle{empty}
\cleardoublepage
\pagestyle{fancy}

\addcontentsline{toc}{chapter}{References}
\bibliographystyle{alpha}
\renewcommand{\bibname}{References}
\bibliography{ref}

\pagestyle{empty}
\cleardoublepage
\pagestyle{fancy}

\appendix
\chapter{Source Code}
\label{appendix:source}
\acresetall


\section{Complete Source Code}
The source code is released in two different ways. The version used to produce
the test results in this thesis is released in a zipped package at the following
address:

\url{https://github.com/espengra/secure-ad-hoc-network-doc/raw/master/share/secure-ad-hoc-network.zip}

For the latest updated version there is a Git repository where you can download
or fork your own branch of the source code from the following address:

\url{https://github.com/espengra/secure-ad-hoc-network}


\section{Code Snippets}
In this section all the code snippets referred to from Chapter
\ref{ch:implementation} are shown. They might not reflect the source code
perfectly, as some lines in between and re-organization has been done to only
include the most important steps. However, all the values are the exact same as
their counterparts in the source code.

\subsection{AM Sockets Setup}\label{code:sockets}
\begin{lstlisting}[frame=tb]
int32_t *recvsock, *sendsock;
addrinfo hints, *res;

/* Set family information */
memset(&hints, 0, sizeof hints);
hints.ai_family = AF_INET;
hints.ai_socktype = SOCK_DGRAM;
hints.ai_flags = AI_PASSIVE;
hints.ai_protocol = IPPROTO_UDP;

/* Puts the port-info inside a addrinfo data structure */
getaddrinfo(NULL, port, &hints, &res);

/* Assign file descriptor for sockets */
*recvsock = socket(PF_INET, SOCK_DGRAM, 0)
*sendsock = socket(PF_INET, SOCK_DGRAM, 0)

/* Binds the sockets to the network interface */
setsockopt(*recvsock, SOL_SOCKET, SO_BINDTODEVICE, interface, strlen(interface) + 1)
setsockopt(*sendsock, SOL_SOCKET, SO_BINDTODEVICE, interface, strlen(interface) + 1)

/* Binds receive socket to the port (rest of the address is empty/null) */
bind(*recvsock, res->ai_addr, res->ai_addrlen);

/* Allow the send socket to send broadcast messages */
int broadcast_val = 1;
setsockopt(*sendsock, SOL_SOCKET, SO_BROADCAST, &broadcast_val, sizeof int)

/* Set the send socket to non-blocking */
fcntl(*sendsock, F_SETFL, O_NONBLOCK);

\end{lstlisting}

\subsection{Proxy Certificate Extension}\label{code:pc_ext}
Remember the author did not get the original proxyCertInfoExtension to work, so
it was changed with a netscape comment.

\begin{lstlisting}[frame=tb]
STACK_OF(X509_EXTENSION) *exts = sk_X509_EXTENSION_new_null();
openssl_cert_add_ext_req(exts, NID_netscape_comment, "critical,myProxyCertInfoExtension:0,1");

X509_REQ_add_extensions(x, exts);
sk_X509_EXTENSION_pop_free(exts, X509_EXTENSION_free);
\end{lstlisting}

The first digit with a 0 in the comment tells the role the node requests is only
regular 'authenticated'. If this was  1 it would mean the node was a SP. The
last digit is used for routing rights. The 1 in the last digit means the node
wishes full routing rights, whereas a 0 would mean limited routing rights.

\subsection{Setting Subject Name in PC}\label{code:set_subject_name}
\begin{lstlisting}[frame=tb]
X509_NAME *name, *req_name, *issuer_name;
req_name = X509_REQ_get_subject_name(req)
issuer_name = X509_get_subject_name(*pc0p)
name = X509_NAME_dup(issuer_name)
req_name_entry = X509_NAME_get_entry(req_name,0);
X509_NAME_add_entry(name, req_name_entry, X509_NAME_entry_count(name), 0);
X509_set_subject_name(cert, name)
\end{lstlisting}

\subsection{Adding Trusted Node to AL}\label{code:add_to_al}
\begin{lstlisting}[frame=tb]
void al_add(uint32_t addr, uint16_t id, role_type role, unsigned char *subject_name, EVP_PKEY *key) {

	authenticated_list[num_auth_nodes] = malloc(sizeof(trusted_node));
	authenticated_list[num_auth_nodes]->addr = addr;
	authenticated_list[num_auth_nodes]->id = id;
	authenticated_list[num_auth_nodes]->role = role;
	authenticated_list[num_auth_nodes]->name = malloc(FULL_SUB_NM_SZ);
	memset(authenticated_list[num_auth_nodes]->name, 0, FULL_SUB_NM_SZ);

	if(strlen((char *)subject_name)>FULL_SUB_NM_SZ)
		memcpy(authenticated_list[num_auth_nodes]->name, subject_name, FULL_SUB_NM_SZ);
	else
		memcpy(authenticated_list[num_auth_nodes]->name, subject_name, strlen((char *)subject_name));

	authenticated_list[num_auth_nodes]->pub_key = openssl_key_copy(key);

	if(id != my_id) {
		EVP_PKEY_free(key);
	}

	num_auth_nodes++;

}
\end{lstlisting}

\subsection{Adding Trusted Neighbor to NL}\label{code:add_to_nl}
\begin{lstlisting}[frame=tb]
void neigh_list_add(uint32_t addr, uint16_t id, unsigned char *mac_value) {

	int i;
	for(i=0; i<num_trusted_neigh; i++) {
		if(id == neigh_list[i]->id) {

			if(addr == neigh_list[i]->addr) {

				if(neigh_list[i]->mac != NULL)
					free(neigh_list[i]->mac);

				neigh_list[i]->mac = mac_value;
				neigh_list[i]->window = 0;
				neigh_list[i]->last_seq_num = 0;
				neigh_list[i]->last_rcvd_time = time (NULL);
				neigh_list[i]->num_keystream_fails = 0;

			} else {

				if (mac_value != NULL)
					free(mac_value);

				neig_list_remove(i);
				
			}
			
			break;
			
		}
		
	}

	if(i==num_trusted_neigh) {

		neigh_list[num_trusted_neigh] = malloc(sizeof(trusted_neigh));
		neigh_list[num_trusted_neigh]->addr = addr;
		neigh_list[num_trusted_neigh]->id = id;
		neigh_list[num_trusted_neigh]->mac = mac_value;
		neigh_list[i]->window = 0;
		neigh_list[num_trusted_neigh]->last_seq_num = 0;
		neigh_list[num_trusted_neigh]->last_rcvd_time = time (NULL);
		neigh_list[num_trusted_neigh]->num_keystream_fails = 0;
		num_trusted_neigh++;

	}

}

\end{lstlisting}

\subsection{Removing Trusted Neighbor to NL}\label{code:rem_from_nl}
\begin{lstlisting}[frame=tb]
int neig_list_remove(int pos) {

	/* First check whether this node exists at all (sanity check) */
	if(neigh_list[pos] == NULL) {
		return 0;
	}

	/* Check whether keystream exists, remove if so! */
	if(neigh_list[pos]->mac != NULL)
		free(neigh_list[pos]->mac);

	/* Free up neighbor in memory */
	free(neigh_list[pos]);

	/* Re-arrange Neighbor List to avoid scarce population */
	int i;
	for(i=pos+1; i<num_trusted_neigh; i++) {
		neigh_list[i-1] = neigh_list[i];
	}

	/* Finally, number of trusted neighbors has shrunk :) */
	num_trusted_neigh--;

	return 1;
}
\end{lstlisting}

\subsection{Generate Ephemeral Key}\label{code:gen_eph_key}
\begin{lstlisting}[frame=tb]
void openssl_key_generate(EVP_CIPHER_CTX *aes_master, int key_count, unsigned char **keyp) {

	unsigned char *ret;
	int i, tmp, ol;

	if(keyp == NULL || *keyp == NULL) {
		ret = malloc(EVP_CIPHER_CTX_block_size(aes_master));
	} else {
		memset(*keyp, 0, EVP_CIPHER_CTX_block_size(aes_master));
		ret = *keyp;
	}

	ol = 0;

	/* Create plaintext from key_count - each new key will be cipher of i=1,2,3... */
	unsigned char *plaintext = malloc(sizeof(key_count));
	memset(plaintext, 0, sizeof(plaintext));
	*plaintext = (unsigned char)key_count;
	int len = strlen((char *)plaintext)+1;

	EVP_EncryptUpdate(aes_master, ret, &tmp, plaintext, len);
	ol += tmp;
	//Remove padding, not wanted for key!
	EVP_EncryptFinal(aes_master, ret, &tmp);

	free(plaintext);
	*keyp = ret;

}
\end{lstlisting}


\subsection{Generate Keystream}\label{code:gen_keystream}
\begin{lstlisting}[frame=tb]
/* Generate Keystream from Nonce */

if(*key_count>1)
	free(auth_value);

int rand_len = RAND_LEN;
auth_value = malloc(rand_len*10+10);
auth_value_len = 0;

for(i=0; i<10; i++) {

	/* Do encryption */
	EVP_CIPHER_CTX current_ctx;
	EVP_EncryptInit(&current_ctx, EVP_aes_128_cbc(), current_key, current_iv);
	unsigned char *tmp = openssl_aes_encrypt(&current_ctx, current_rand, &value_len);
	EVP_CIPHER_CTX_cleanup(&current_ctx);

	/* Place ciphertext in keystream */
	int auth_pos = auth_value_len;
	auth_value_len += value_len;
	memcpy(auth_value+auth_pos, tmp, value_len);

	/* Change to new IV */
	memcpy(current_iv, tmp, AES_IV_SIZE);

	/* Alter the Nonce before next encryption */
	int j;
	for(j=0;j<rand_len/10; j++) {
		current_rand[j+(i*(rand_len/10))] = ( (current_rand[j+(i*(rand_len/10))]) ^ i );
	}
	
	free(tmp);
	value_len = RAND_LEN;

}
\end{lstlisting}

\subsection{Extension in BATMAN Class}\label{code:ext_batman}
\begin{lstlisting}[frame=tb]
/******************** Begin Authentication Module Extension ********************/

/*
 * If the daemon is not authenticated, or it receives an authentication
 * token which it does not recognize, the authentication procedure in the
 * Authentication Module is called. No packets received when authenticating
 * will be processed.
 */

if(num_trusted_neigh) {
	for(neigh_counter = 0; neigh_counter < num_trusted_neigh; neigh_counter++) {
		if(neigh_list[neigh_counter]->addr == neigh) {
			break;
		}
	}
}

if(neigh_counter == num_trusted_neigh) {

	if(my_role == SP && my_state ==  READY) {
		new_neighbor = neigh;
	}

	if(my_role == AUTHENTICATED && my_state ==  READY) {
		/* Check to see whether the other node is AUTHENTICATED */
		if(memcmp(&(bat_packet->auth), empty_check, 2) != 0)
			new_neighbor = neigh;
	}

	goto send_packets;
}

if(neigh_list[neigh_counter]->mac == NULL)
	goto send_packets;

if(memcmp(neigh_list[neigh_counter]->mac+(bat_packet->auth_seqno*2), bat_packet->auth, 2) != 0) {

	printf("MAC Extract did not match!\n");

	if(my_state == READY) {

		neigh_list[neigh_counter]->num_keystream_fails ++;

		/* Keystream is consequently fail, ergo need to handshake a new one */
		if(neigh_list[neigh_counter]->num_keystream_fails > 20) {
			my_state = WAIT_FOR_REQ_SIG;
			new_neighbor = neigh;
			neigh_list[neigh_counter]->num_keystream_fails = 0;
		}

	}

	goto send_packets;
}

/* Check whether the packet is new and not a replayed packet */
if(!tool_sliding_window(bat_packet->auth_seqno, neigh_list[neigh_counter]->id))
	goto send_packets;


/* Everything seems fine, reset failcounter if more than 0 */
if(neigh_list[neigh_counter]->num_keystream_fails != 0)
	neigh_list[neigh_counter]->num_keystream_fails = 0;

/********************* End Authentication Module Extension *********************/
\end{lstlisting}

\subsection{Extension in SCHEDULE Class}\label{code:ext_schedule}
\begin{lstlisting}[frame=tb]
/* Begin Authentication Module Extension */

/* Add Signature Extract to OGM */
if(pthread_mutex_trylock(&auth_lock) == 0) {

	if(auth_value != NULL) {

		memcpy(bat_packet->auth, auth_value+2*auth_seq_num, 2);
		bat_packet->auth_seqno = auth_seq_num;
		auth_seq_num ++;

	}

	pthread_mutex_unlock(&auth_lock);
}

/* End Authentication Module Extension */
\end{lstlisting}

%\section{BATMAN}
%
%\subsection{batman.h - struct bat\_packet}
%\lstinputlisting[frame=tb]{source_code/batman.h}
%
%\subsection{batman.c - batman()}
%\lstinputlisting[frame=tb]{source_code/batman.c}
%
%
%\section{SCHEDULE}
%
%\subsection{schedule.c - excerpt}
%Line numbers indicate where the code is added to the original source code.
%\lstinputlisting[frame=tb]{source_code/schedule.c}
%
%
%\section{AM}
%
%\subsection{am.h}
%\lstinputlisting[frame=tb]{source_code/am.h}
%
%\subsection{am.c}
%\lstinputlisting[frame=tb]{source_code/am.c}


\pagestyle{empty}
\cleardoublepage
\pagestyle{fancy}

\chapter{IRC Chat Logs}
\label{appendix:irc}
\acresetall

\definecolor{darkgray}{rgb}{0.95,0.95,0.95}
\lstset{
	language=c,
	basicstyle=\footnotesize,
	numbers=left,
	numberstyle=\footnotesize,
	stepnumber=0,
	numbersep=5pt,
	backgroundcolor=\color{darkgray},
	showspaces=false,
	showstringspaces=false,
	showtabs=false,
	frame=single,
	tabsize=2,
	captionpos=b,
	breaklines=true,
	breakatwhitespace=false,
	escapeinside={\%*}{*)}
}


\section{Previous Sender Field}
%\lstinputlisting[frame=tb]{irc_logs/03_26_2011_#batman.txt}

%\section{BATMAN}
%
%\subsection{batman.h - struct bat\_packet}
%\lstinputlisting[frame=tb]{source_code/batman.h}
%
%\subsection{batman.c - batman()}
%\lstinputlisting[frame=tb]{source_code/batman.c}
%
%
%\section{SCHEDULE}
%
%\subsection{schedule.c - excerpt}
%Line numbers indicate where the code is added to the original source code.
%\lstinputlisting[frame=tb]{source_code/schedule.c}
%
%
%\section{AM}
%
%\subsection{am.h}
%\lstinputlisting[frame=tb]{source_code/am.h}
%
%\subsection{am.c}
%\lstinputlisting[frame=tb]{source_code/am.c}


\pagestyle{empty}
\cleardoublepage
\pagestyle{fancy}

\chapter{Lab Setup}
\label{lab_setup}
\definecolor{darkgray}{rgb}{0.95,0.95,0.95}
\lstset{ %
language=c,                % choose the language of the code
basicstyle=\footnotesize,       % the size of the fonts that are used for the code
numbers=left,                   % where to put the line-numbers
numberstyle=\footnotesize,      % the size of the fonts that are used for the line-numbers
stepnumber=0,                   % the step between two line-numbers. If it is 1 each line will be numbered
numbersep=5pt,                  % how far the line-numbers are from the code
backgroundcolor=\color{darkgray},  % choose the background color. You must add \usepackage{color}
showspaces=false,               % show spaces adding particular underscores
showstringspaces=false,         % underline spaces within strings
showtabs=false,                 % show tabs within strings adding particular underscores
frame=single,   		% adds a frame around the code
tabsize=2,  		% sets default tabsize to 2 spaces
captionpos=b,   		% sets the caption-position to bottom
breaklines=true,    	% sets automatic line breaking
breakatwhitespace=false,    % sets if automatic breaks should only happen at whitespace
escapeinside={\%*}{*)}          % if you want to add a comment within your code
}

The computers used in the lab was setup with the following hardware:

\begin{itemize}
\item Intel Core 2 Duo 2.83 GHz processor
\item 4 GB memory
\item Atheros AR5413 802.11abg NIC
\end{itemize}

\noindent
Further, they are setup with Ubuntu 10.4 (Linux Kernel 2.6.32-25-generic-pae) and ath5k drivers for the wireless interfaces. The network interface is configured as follows:


\begin{lstlisting}[frame=tb]
/etc/network/interfaces

auto lo
iface lo inet loopback

auto wlan0
iface wlan0 inet static
address 10.0.0.X
netmask 255.255.255.0
pre-up ifconfig wlan0 down
pre-up ifconfig wlan0 hw ether XX:XX:XX:XX:XX:XX
pre-up iwconfig wlan0 mode ad-hoc essid BATMAN channel 3

auto unicast
iface unicast inet static
address 10.0.0.X
netmask 255.255.255.0
pre-up brctl addbr unicast
pre-up brctl addif unicast wlan0
pre-down ifconfig unicast down
post-down brctl delif unicast wlan0
post-down brctl delbr unicast
\end{lstlisting}

\noindent
\\
To install batmand, run the following as root user:

\begin{lstlisting}[frame=tb]
make
make install
make clean
\end{lstlisting}

\noindent
\\
For running the batman daemon on the test nodes, we ran the following script as root user:

\begin{lstlisting}[frame=tb]
ifconfig wlan0 up
batmand wlan0
batmand -cd 4
killall batmand
ifconfig wlan0 down
\end{lstlisting}

\noindent
\\
For reducing the transmitting power in test III and IV we ran the following as root user:

\begin{lstlisting}[frame=tb]
ifconfig wlan0 down
iwconfig wlan0 txpower 7
ifconfig wlan0 up
\end{lstlisting}




\end{document}